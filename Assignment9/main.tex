\documentclass[journal,12pt,twocolumn]{IEEEtran}

\usepackage{setspace}
\usepackage{gensymb}
\singlespacing
\usepackage[cmex10]{amsmath}

\usepackage{amsthm}

\usepackage{mathrsfs}
\usepackage{txfonts}
\usepackage{stfloats}
\usepackage{bm}
\usepackage{cite}
\usepackage{cases}
\usepackage{subfig}

\usepackage{longtable}
\usepackage{multirow}

\usepackage{enumitem}
\usepackage{mathtools}
\usepackage{steinmetz}
\usepackage{tikz}
\usepackage{circuitikz}
\usepackage{verbatim}
\usepackage{tfrupee}
\usepackage[breaklinks=true]{hyperref}
\usepackage{graphicx}
\usepackage{tkz-euclide}

\usetikzlibrary{calc,math}
\usepackage{listings}
    \usepackage{color}                                            %%
    \usepackage{array}                                            %%
    \usepackage{longtable}                                        %%
    \usepackage{calc}                                             %%
    \usepackage{multirow}                                         %%
    \usepackage{hhline}                                           %%
    \usepackage{ifthen}                                           %%
    \usepackage{lscape}     
\usepackage{multicol}
\usepackage{chngcntr}

\DeclareMathOperator*{\Res}{Res}

\renewcommand\thesection{\arabic{section}}
\renewcommand\thesubsection{\thesection.\arabic{subsection}}
\renewcommand\thesubsubsection{\thesubsection.\arabic{subsubsection}}

\renewcommand\thesectiondis{\arabic{section}}
\renewcommand\thesubsectiondis{\thesectiondis.\arabic{subsection}}
\renewcommand\thesubsubsectiondis{\thesubsectiondis.\arabic{subsubsection}}


\hyphenation{op-tical net-works semi-conduc-tor}
\def\inputGnumericTable{}                                 %%

\lstset{
%language=C,
frame=single, 
breaklines=true,
columns=fullflexible
}
\begin{document}


\newtheorem{theorem}{Theorem}[section]
\newtheorem{problem}{Problem}
\newtheorem{proposition}{Proposition}[section]
\newtheorem{lemma}{Lemma}[section]
\newtheorem{corollary}[theorem]{Corollary}
\newtheorem{example}{Example}[section]
\newtheorem{definition}[problem]{Definition}

\newcommand{\BEQA}{\begin{eqnarray}}
\newcommand{\EEQA}{\end{eqnarray}}
\newcommand{\define}{\stackrel{\triangle}{=}}
\bibliographystyle{IEEEtran}
\raggedbottom
\setlength{\parindent}{0pt}
\providecommand{\mbf}{\mathbf}
\providecommand{\pr}[1]{\ensuremath{\Pr\left(#1\right)}}
\providecommand{\qfunc}[1]{\ensuremath{Q\left(#1\right)}}
\providecommand{\sbrak}[1]{\ensuremath{{}\left[#1\right]}}
\providecommand{\lsbrak}[1]{\ensuremath{{}\left[#1\right.}}
\providecommand{\rsbrak}[1]{\ensuremath{{}\left.#1\right]}}
\providecommand{\brak}[1]{\ensuremath{\left(#1\right)}}
\providecommand{\lbrak}[1]{\ensuremath{\left(#1\right.}}
\providecommand{\rbrak}[1]{\ensuremath{\left.#1\right)}}
\providecommand{\cbrak}[1]{\ensuremath{\left\{#1\right\}}}
\providecommand{\lcbrak}[1]{\ensuremath{\left\{#1\right.}}
\providecommand{\rcbrak}[1]{\ensuremath{\left.#1\right\}}}
\theoremstyle{remark}
\newtheorem{rem}{Remark}
\newcommand{\sgn}{\mathop{\mathrm{sgn}}}
\providecommand{\abs}[1]{\left\vert#1\right\vert}
\providecommand{\res}[1]{\Res\displaylimits_{#1}} 
\providecommand{\norm}[1]{\left\lVert#1\right\rVert}
%\providecommand{\norm}[1]{\lVert#1\rVert}
\providecommand{\mtx}[1]{\mathbf{#1}}
\providecommand{\mean}[1]{E\left[ #1 \right]}
\providecommand{\fourier}{\overset{\mathcal{F}}{ \rightleftharpoons}}
%\providecommand{\hilbert}{\overset{\mathcal{H}}{ \rightleftharpoons}}
\providecommand{\system}{\overset{\mathcal{H}}{ \longleftrightarrow}}
	%\newcommand{\solution}[2]{\textbf{Solution:}{#1}}
\newcommand{\solution}{\noindent \textbf{Solution: }}
\newcommand{\cosec}{\,\text{cosec}\,}
\providecommand{\dec}[2]{\ensuremath{\overset{#1}{\underset{#2}{\gtrless}}}}
\newcommand{\myvec}[1]{\ensuremath{\begin{pmatrix}#1\end{pmatrix}}}
\newcommand{\mydet}[1]{\ensuremath{\begin{vmatrix}#1\end{vmatrix}}}
\numberwithin{equation}{subsection}
\makeatletter
\@addtoreset{figure}{problem}
\makeatother
\let\StandardTheFigure\thefigure
\let\vec\mathbf
\renewcommand{\thefigure}{\theproblem}
\def\putbox#1#2#3{\makebox[0in][l]{\makebox[#1][l]{}\raisebox{\baselineskip}[0in][0in]{\raisebox{#2}[0in][0in]{#3}}}}
     \def\rightbox#1{\makebox[0in][r]{#1}}
     \def\centbox#1{\makebox[0in]{#1}}
     \def\topbox#1{\raisebox{-\baselineskip}[0in][0in]{#1}}
     \def\midbox#1{\raisebox{-0.5\baselineskip}[0in][0in]{#1}}
\vspace{3cm}
\title{Assignment 9}
\author{Pulkit Saxena}
\maketitle
\newpage
%\tableofcontents
\bigskip
\renewcommand{\thefigure}{\theenumi}
\renewcommand{\thetable}{\theenumi}

%
\section{Question Hoffman PG26 Q1 }
Let
\begin{align}
\vec{A}=\myvec{1&2&1&0\\-1&0&3&5\\1&-2&1&1}  
\end{align}
Find a row-reduced echelon matrix $\vec{R}$ which is row-equivalent to $\vec{A}$ and an invertible 3x3 matrix $\vec{P}$ such that $\vec{R}$ = $\vec{P}$ $\vec{A}$.
\section{\textbf{Solution}}Given 
\begin{align}
\vec{A}=\myvec{1&2&1&0\\-1&0&3&5\\1&-2&1&1}
\end{align}
Row reduce $\vec{A}$ by applying the elementary row operations and equivalently at each operations find the elementary matrix $\vec{E}$
\begin{align}
    \vec{A|I}=&\myvec{1&2&1&0&\vrule&1&0&0\\-1&0&3&5&\vrule&0&1&0\\1&-2&1&1&\vrule&0&0&1}
\end{align}
\begin{multline}
    \xleftrightarrow{R_2=R_2+R_1}\myvec{1&2&1&0&\vrule&1&0&0\\0&2&4&5&\vrule&1&1&0\\1&-2&1&1&\vrule&0&0&1}\\
    \implies\vec{e_1}=\myvec{1&0&0\\1&1&0\\0&0&1} 
\end{multline}
\begin{multline}
\xleftrightarrow{R_3=R_3-R_1}\myvec{1&2&1&0&\vrule&1&0&0\\0&2&4&5&\vrule&1&1&0\\0&-4&0&1&\vrule&-1&0&1}\\
\implies\vec{e_2}=\myvec{1&0&0\\1&1&0\\-1&0&1}
\end{multline}
\begin{multline}
\xleftrightarrow{R_1=R_1-R_2}\myvec{1&0&-3&-5&\vrule&0&-1&0\\0&2&4&5&\vrule&1&1&0\\0&-4&0&1&\vrule&-1&0&1}\\
\implies\vec{e_3}=\myvec{0&-1&0\\1&1&0\\-1&0&1}
\end{multline}
\begin{multline}
\xleftrightarrow{R_3=R_3+2R_2}\myvec{1&0&-3&-5&\vrule&0&-1&0\\0&2&4&5&\vrule&1&1&0\\0&0&8&11&\vrule&1&2&1}\\
\implies\vec{e_4}=\myvec{0&-1&0\\1&1&0\\1&2&1}
\end{multline}
\begin{multline}
\xleftrightarrow{R_2=\frac{R_2}{2}}\myvec{1&0&-3&-5&\vrule&0&-1&0\\0&1&2&\frac{5}{2}&\vrule&\frac{1}{2}&\frac{1}{2}&0\\0&0&8&11&\vrule&1&2&1}\\
\implies\vec{e_5}=\myvec{0&-1&0\\\frac{1}{2}&\frac{1}{2}&0\\1&2&1}
\end{multline}
\begin{multline}
\xleftrightarrow{R_3=\frac{R_3}{8}}\myvec{1&0&-3&-5&\vrule&0&-1&0\\0&1&2&\frac{5}{2}&\vrule&\frac{1}{2}&\frac{1}{2}&0\\0&0&1&\frac{11}{8}&\vrule&\frac{1}{8}&\frac{1}{4}&\frac{1}{8}}\\
\implies\vec{e_6}=\myvec{0&-1&0\\\frac{1}{2}&\frac{1}{2}&0\\\frac{1}{8}&\frac{1}{4}&\frac{1}{8}}
\end{multline}
\begin{multline}
\xleftrightarrow{R_1=R_1+3R_3}\myvec{1&0&0&-\frac{7}{8}&\vrule&\frac{3}{8}&-\frac{1}{4}&\frac{3}{8}\\0&1&2&\frac{5}{2}&\vrule&\frac{1}{2}&\frac{1}{2}&0\\0&0&1&\frac{11}{8}&\vrule&\frac{1}{8}&\frac{1}{4}&\frac{1}{8}}\\
\implies\vec{e_7}=\myvec{\frac{3}{8}&-\frac{1}{4}&\frac{3}{8}\\\frac{1}{2}&\frac{1}{2}&0\\\frac{1}{8}&\frac{1}{4}&\frac{1}{8}}
\end{multline}
\begin{multline}
\xleftrightarrow{R_2=R_2-2R_3}\myvec{1&0&0&-\frac{7}{8}&\vrule&\frac{3}{8}&-\frac{1}{4}&\frac{3}{8}\\0&1&0&-\frac{1}{4}&\vrule&\frac{1}{4}&0&-\frac{1}{4}\\0&0&1&\frac{11}{8}&\vrule&\frac{1}{8}&\frac{1}{4}&\frac{1}{8}}\\
\implies\vec{e_8}=\myvec{\frac{3}{8}&-\frac{1}{4}&\frac{3}{8}\\\frac{1}{4}&0&-\frac{1}{4}\\\frac{1}{8}&\frac{1}{4}&\frac{1}{8}}\label{1}
\end{multline}
Hence,row reduced echelon matrix that is row equivalent to $\vec{A}$ is
\begin{align}
\vec{R}=\myvec{1&0&0&-\frac{7}{8}\\0&1&0&-\frac{1}{4}\\0&0&1&\frac{11}{8}}
\end{align}
where,
\begin{align}
    \Vec{E}=\vec{e_1}\vec{e_2}\vec{e_3}\vec{e_4}\vec{e_5}\vec{e_6}\vec{e_7}\vec{e_8}
\end{align}
are the elementary matrices that transform $\vec{A}$ to $\vec{R}$
\begin{align}
  \vec{e_1}\vec{e_2}\vec{e_3}\vec{e_4}\vec{e_5}\vec{e_6}\vec{e_7}\vec{e_8}\vec{A}=\vec{R}\implies\vec{P}=\vec{e_1}\vec{e_2}\vec{e_3}\vec{e_4}\vec{e_5}\vec{e_6}\vec{e_7}\vec{e_8}  
\end{align}
Since elementary matrices are invertible and the product of invetible matrices is invertible ,thus
\begin{align}
  \vec{P}=\vec{e_1}\vec{e_2}\vec{e_3}\vec{e_4}\vec{e_5}\vec{e_6}\vec{e_7}\vec{e_8}  
\end{align}
is invertible.\\
From \eqref{1}
\begin{align}
\vec{P}=\myvec{\frac{3}{8}&-\frac{1}{4}&\frac{3}{8}\\\frac{1}{4}&0&-\frac{1}{4}\\\frac{1}{8}&\frac{1}{4}&\frac{1}{8}}\\
\vec{R}=\myvec{1&0&0&-\frac{7}{8}\\0&1&0&-\frac{1}{4}\\0&0&1&\frac{11}{8}}
\end{align}
such that $\vec{R}=\vec{P}\vec{A}$.
\end{document}
