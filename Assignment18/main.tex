\documentclass[journal,12pt]{IEEEtran}
\usepackage{longtable}
\usepackage{setspace}
\usepackage{gensymb}
\singlespacing
\usepackage[cmex10]{amsmath}
\newcommand\myemptypage{
	\null
	\thispagestyle{empty}
	\addtocounter{page}{-1}
	\newpage
}
\usepackage{amsthm}
\usepackage{mdframed}
\usepackage{mathrsfs}
\usepackage{txfonts}
\usepackage{stfloats}
\usepackage{bm}
\usepackage{cite}
\usepackage{cases}
\usepackage{subfig}

\usepackage{longtable}
\usepackage{multirow}

\usepackage{enumitem}
\usepackage{mathtools}
\usepackage{steinmetz}
\usepackage{tikz}
\usepackage{circuitikz}
\usepackage{verbatim}
\usepackage{tfrupee}
\usepackage[breaklinks=true]{hyperref}
\usepackage{graphicx}
\usepackage{tkz-euclide}

\usetikzlibrary{calc,math}
\usepackage{listings}
    \usepackage{color}                                            %%
    \usepackage{array}                                            %%
    \usepackage{longtable}                                        %%
    \usepackage{calc}                                             %%
    \usepackage{multirow}                                         %%
    \usepackage{hhline}                                           %%
    \usepackage{ifthen}                                           %%
    \usepackage{lscape}     
\usepackage{multicol}
\usepackage{chngcntr}

\DeclareMathOperator*{\Res}{Res}

\renewcommand\thesection{\arabic{section}}
\renewcommand\thesubsection{\thesection.\arabic{subsection}}
\renewcommand\thesubsubsection{\thesubsection.\arabic{subsubsection}}

\renewcommand\thesectiondis{\arabic{section}}
\renewcommand\thesubsectiondis{\thesectiondis.\arabic{subsection}}
\renewcommand\thesubsubsectiondis{\thesubsectiondis.\arabic{subsubsection}}


\hyphenation{op-tical net-works semi-conduc-tor}
\def\inputGnumericTable{}                                 %%

\lstset{
%language=C,
frame=single, 
breaklines=true,
columns=fullflexible
}
\begin{document}
\onecolumn

\newtheorem{theorem}{Theorem}[section]
\newtheorem{problem}{Problem}
\newtheorem{proposition}{Proposition}[section]
\newtheorem{lemma}{Lemma}[section]
\newtheorem{corollary}[theorem]{Corollary}
\newtheorem{example}{Example}[section]
\newtheorem{definition}[problem]{Definition}

\newcommand{\BEQA}{\begin{eqnarray}}
\newcommand{\EEQA}{\end{eqnarray}}
\newcommand{\define}{\stackrel{\triangle}{=}}
\bibliographystyle{IEEEtran}
\raggedbottom
\setlength{\parindent}{0pt}
\providecommand{\mbf}{\mathbf}
\providecommand{\pr}[1]{\ensuremath{\Pr\left(#1\right)}}
\providecommand{\qfunc}[1]{\ensuremath{Q\left(#1\right)}}
\providecommand{\sbrak}[1]{\ensuremath{{}\left[#1\right]}}
\providecommand{\lsbrak}[1]{\ensuremath{{}\left[#1\right.}}
\providecommand{\rsbrak}[1]{\ensuremath{{}\left.#1\right]}}
\providecommand{\brak}[1]{\ensuremath{\left(#1\right)}}
\providecommand{\lbrak}[1]{\ensuremath{\left(#1\right.}}
\providecommand{\rbrak}[1]{\ensuremath{\left.#1\right)}}
\providecommand{\cbrak}[1]{\ensuremath{\left\{#1\right\}}}
\providecommand{\lcbrak}[1]{\ensuremath{\left\{#1\right.}}
\providecommand{\rcbrak}[1]{\ensuremath{\left.#1\right\}}}
\theoremstyle{remark}
\newtheorem{rem}{Remark}
\newcommand{\sgn}{\mathop{\mathrm{sgn}}}
\providecommand{\abs}[1]{\left\vert#1\right\vert}
\providecommand{\res}[1]{\Res\displaylimits_{#1}} 
\providecommand{\norm}[1]{\left\lVert#1\right\rVert}
%\providecommand{\norm}[1]{\lVert#1\rVert}
\providecommand{\mtx}[1]{\mathbf{#1}}
\providecommand{\mean}[1]{E\left[ #1 \right]}
\providecommand{\fourier}{\overset{\mathcal{F}}{ \rightleftharpoons}}
%\providecommand{\hilbert}{\overset{\mathcal{H}}{ \rightleftharpoons}}
\providecommand{\system}{\overset{\mathcal{H}}{ \longleftrightarrow}}
	%\newcommand{\solution}[2]{\textbf{Solution:}{#1}}
\newcommand{\solution}{\noindent \textbf{Solution: }}
\newcommand{\cosec}{\,\text{cosec}\,}
\providecommand{\dec}[2]{\ensuremath{\overset{#1}{\underset{#2}{\gtrless}}}}
\newcommand{\myvec}[1]{\ensuremath{\begin{pmatrix}#1\end{pmatrix}}}
\newcommand{\mydet}[1]{\ensuremath{\begin{vmatrix}#1\end{vmatrix}}}
\numberwithin{equation}{subsection}
\makeatletter
\@addtoreset{figure}{problem}
\makeatother
\let\StandardTheFigure\thefigure
\let\vec\mathbf
\renewcommand{\thefigure}{\theproblem}
\def\putbox#1#2#3{\makebox[0in][l]{\makebox[#1][l]{}\raisebox{\baselineskip}[0in][0in]{\raisebox{#2}[0in][0in]{#3}}}}
     \def\rightbox#1{\makebox[0in][r]{#1}}
     \def\centbox#1{\makebox[0in]{#1}}
     \def\topbox#1{\raisebox{-\baselineskip}[0in][0in]{#1}}
     \def\midbox#1{\raisebox{-0.5\baselineskip}[0in][0in]{#1}}
\vspace{3cm}
\title{Assignment 18}
\author{Pulkit Saxena}
\maketitle
\renewcommand{\thefigure}{\theenumi}
\renewcommand{\thetable}{\theenumi}
\section{\textbf{Problem UGC JUNE 2015 Q71}}
Let $S$ be the set of 3x3 real matrices $\vec{A}$ with 
\begin{align}
    \vec{A}^T\vec{A}=\myvec{1&0&0\\0&0&0\\0&0&0}
\end{align}
Then the set contains:-\\
\begin{enumerate}
    \item a Nilpotent Matrix
    \item a matrix of rank one
    \item a matrix of rank two
    \item a non-zero skew symmetric matrix.
\end{enumerate}
\section{\textbf{Essential Framework Required to solve problem }}
\renewcommand{\thetable}{1}
\begin{longtable}{|l|l|}
	\hline
	\multirow{3}{*}{Proof 1}	& \\
	&Let $\vec{A}x$=0 and $\mathbb{N(\vec{A})}$ is the null space of $\vec{A}$\\
	&\\
	$Rank(\vec{A})=Rank(\vec{A}^T\vec{A})$&Then $\vec{A}^T\vec{A}$x=0 which means $\mathbb{N(\vec{A})}\subset \mathbb{N(\vec{A}^T\vec{A})}$ \\
	&\\
	&Thus if $\vec{A}^T\vec{A}$x=0 ,then\\
	&\\
	&$x^T\vec{A}^T\vec{A}x=0\implies\lVert\vec{A}x\rVert=0$\\
	&\\
	&Which means $\vec{A}x=0$ thus\\
	&\\
	& $\mathbb{N(\vec{A}^T\vec{A})}\subset\mathbb{N(\vec{A})}$\\
	&\\
	&From the Above two condition we can say that ${N(\vec{A}^T\vec{A})}=\mathbb{N(\vec{A})}$\\
	&\\
	&$rank(\vec{A})=n-\mathbb{N(\vec{A})}$\\
	&\\
	&$rank(\vec{A})=rank(\vec{A}^T\vec{A})$\\
	&\\
	&Hence Proved.\\
	&\\
	\hline
	\multirow{3}{*}{Proof 2} 
	&\\
	&Suppose $\vec{A}=\myvec{\vec{a_1}&\hdots&\vec{a_n}}$ where $\vec{a_i}$ is the column vector of $\vec{A}$\\
	&\\
	Rowspace$(\vec{A}^T\vec{A})$=Rowspace($\vec{A}$) & $\vec{A}^T\vec{A}=\vec{A}^T\myvec{\vec{a_1}&\hdots&\vec{a_n}}=\myvec{\vec{A}^T\vec{a_1}&\hdots\vec{A}^T\vec{a_n}}$\\
	&\\
	&For each column of $\vec{A}^T\vec{A}$\\
	&\\
	&$\vec{A}^T\vec{a_i}=\myvec{\vec{b_1}&\hdots\vec{b_n}}\vec{a_i}$where $\vec{b_i}$ is the column vector of $\vec{A}^T$ and Row of $\vec{A}$\\
	&\\
	&$=\myvec{\vec{b_1}&\hdots\vec{b_n}}\myvec{a_{i1}\\ \vdots \\a_{in}}=\sum_{j=1}^{n}a_{ij}b_j$\\
	&\\
	&So column of $\vec{A}^T\vec{A}$ is the linear combination of rows of $\vec{A}$.\\
	&\\
	&Since rank$(\vec{A}^T)$=rank$(\vec{A})$ so,\\
	&\\
	&Row$(\vec{A}^T\vec{A})=$Column$(\vec{A}^T\vec{A})$=Row$(\vec{A})$\\
	&\\
	&Hence Proved.\\

	&\\
\hline
  
    \caption{Proofs}
    \label{table:1}
\end{longtable}
\section{\textbf{Solution}}
\renewcommand{\thetable}{2}
\begin{longtable}{|l|l|}
	\hline
	\multirow{3}{*}{Option 1} & \\
	&From Proof 2,Set $S$ contained a set of matrix whose First Column is Non-zero. \\ 
    & \\
    Nilpotent Matrix check&$S\in$ Set$\myvec{1&0&0\\0&0&0\\0&0&0}$,$\myvec{0&0&0\\1&0&0\\0&0&0}$,$\myvec{0&0&0\\0&0&0\\1&0&0}$\\
    &\\
    &Given $\vec{A}^T\vec{A}=\myvec{1&0&0\\0&0&0\\0&0&0}$\\
    &\\
    &So the only matrix $\vec{A}$ which satisfy $\vec{A}^T\vec{A}=\myvec{1&0&0\\0&0&0\\0&0&0}$, $\vec{A}^2=0$ such that $\vec{A}\in S$\\
    &\\
    &$\vec{A}=\myvec{0&0&0\\1&0&0\\0&0&0}\in S$\\
    &\\
    &$\vec{A}^T\vec{A}=\myvec{0&1&0\\0&0&0\\0&0&0}\myvec{0&0&0\\1&0&0\\0&0&0}=\myvec{1&0&0\\0&0&0\\0&0&0}$\\
    
    &\\
    &$\vec{A}^2=\myvec{0&0&0\\1&0&0\\0&0&0}\myvec{0&0&0\\1&0&0\\0&0&0}=\myvec{0&0&0\\0&0&0\\0&0&0}$ which is a nilpotent matrix\\
    &\\
    &Option 1 is correct.\\
    &\\
    \hline
	\multirow{3}{*}{Option 2}
	& \\
    &In Proof 1 we already prove that $Rank(\vec{A})=Rank(\vec{A}^T\vec{A})$\\
    &\\
    matrix of rank one check &Since the $Rank(\vec{A}^T\vec{A})=1$ so the $Rank(\vec{A})=1$ \\ 
	&\\
	&There fore Set S always contains only Rank 1 matrices.\\
	&\\
	&Hence Option 2 is correct.\\
	&\\
	\hline
	\multirow{3}{*}{Option 3}
	&\\
    &Since set S contain only rank 1 matrices and none of rank 2 matrices \\
    &\\
    matrix of rank two check&as already proved above therefore\\
    &\\
    &Option 3 is incorrect.\\
    &\\
    
    \hline
	\multirow{3}{*}{Option 4}
	&\\
	&Proved by contradiction\\
	&\\
    non-zero skew .&Assume Rank of $\vec{A}$ is 1 so $\vec{A}$ can be written as $\vec{A}=\vec{u}\vec{v}^T$ for any non-zero\\
    &\\
    symmetric matrix check&Columns vectors $\vec{u}$ , $\vec{v}$ with n entries. If A is skew symmetric,we have:-\\
    &\\
    &$\vec{A}^T=-\vec{A}$\\
    &\\
    &$(\vec{u}\vec{v})^T=-\vec{u}\vec{v}^T$ $\vec{v}\vec{u}^T=-\vec{u}\vec{v}^T$\\
    &\\
    &The Column space of these matrices is same.The column space of $\vec{v}\vec{u}^T$\\ 
    &is span of $\vec{v}$,where as the column space of $\vec{u}\vec{v}^T$ is the span of $\vec{u}$,\\
    &\\
    &So we must have $\vec{v}=k\vec{u}$ for some $k\in\mathbb{R}$.So the equation becomes\\
    &\\
    &$k\vec{u}\vec{u}^T=-k\vec{u}\vec{u}^T$ \\
    &\\
    &and since $\vec{u}\neq 0$;We can conclude that k=0,which means $\vec{v}=0$ therefore $\vec{A}=0$.\\
    &\\
    &This Contradicts our assumption that $\vec{A}$has rank 1.\\
    &\\
    &Thus real skew symmentric matrix can never have rank=1.\\
    &\\
    &Hence option 4 is incorrect.\\
    &\\
	\hline
	\multirow{3}{*}{Answers}
	&\\
&Option 1 and Option 2 are correct.\\
&\\
	\hline
	
	\caption{Solution Table}
    \label{table:2}
\end{longtable}
\end{document}
