\documentclass[journal,12pt]{IEEEtran}
\usepackage{longtable}
\usepackage{setspace}
\usepackage{gensymb}
\singlespacing
\usepackage[cmex10]{amsmath}
\newcommand\myemptypage{
	\null
	\thispagestyle{empty}
	\addtocounter{page}{-1}
	\newpage
}
\usepackage{amsthm}
\usepackage{mdframed}
\usepackage{mathrsfs}
\usepackage{txfonts}
\usepackage{stfloats}
\usepackage{bm}
\usepackage{cite}
\usepackage{cases}
\usepackage{subfig}

\usepackage{longtable}
\usepackage{multirow}

\usepackage{enumitem}
\usepackage{mathtools}
\usepackage{steinmetz}
\usepackage{tikz}
\usepackage{circuitikz}
\usepackage{verbatim}
\usepackage{tfrupee}
\usepackage[breaklinks=true]{hyperref}
\usepackage{graphicx}
\usepackage{tkz-euclide}

\usetikzlibrary{calc,math}
\usepackage{listings}
    \usepackage{color}                                            %%
    \usepackage{array}                                            %%
    \usepackage{longtable}                                        %%
    \usepackage{calc}                                             %%
    \usepackage{multirow}                                         %%
    \usepackage{hhline}                                           %%
    \usepackage{ifthen}                                           %%
    \usepackage{lscape}     
\usepackage{multicol}
\usepackage{chngcntr}

\DeclareMathOperator*{\Res}{Res}

\renewcommand\thesection{\arabic{section}}
\renewcommand\thesubsection{\thesection.\arabic{subsection}}
\renewcommand\thesubsubsection{\thesubsection.\arabic{subsubsection}}

\renewcommand\thesectiondis{\arabic{section}}
\renewcommand\thesubsectiondis{\thesectiondis.\arabic{subsection}}
\renewcommand\thesubsubsectiondis{\thesubsectiondis.\arabic{subsubsection}}


\hyphenation{op-tical net-works semi-conduc-tor}
\def\inputGnumericTable{}                                 %%

\lstset{
%language=C,
frame=single, 
breaklines=true,
columns=fullflexible
}
\begin{document}
\onecolumn

\newtheorem{theorem}{Theorem}[section]
\newtheorem{problem}{Problem}
\newtheorem{proposition}{Proposition}[section]
\newtheorem{lemma}{Lemma}[section]
\newtheorem{corollary}[theorem]{Corollary}
\newtheorem{example}{Example}[section]
\newtheorem{definition}[problem]{Definition}

\newcommand{\BEQA}{\begin{eqnarray}}
\newcommand{\EEQA}{\end{eqnarray}}
\newcommand{\define}{\stackrel{\triangle}{=}}
\bibliographystyle{IEEEtran}
\raggedbottom
\setlength{\parindent}{0pt}
\providecommand{\mbf}{\mathbf}
\providecommand{\pr}[1]{\ensuremath{\Pr\left(#1\right)}}
\providecommand{\qfunc}[1]{\ensuremath{Q\left(#1\right)}}
\providecommand{\sbrak}[1]{\ensuremath{{}\left[#1\right]}}
\providecommand{\lsbrak}[1]{\ensuremath{{}\left[#1\right.}}
\providecommand{\rsbrak}[1]{\ensuremath{{}\left.#1\right]}}
\providecommand{\brak}[1]{\ensuremath{\left(#1\right)}}
\providecommand{\lbrak}[1]{\ensuremath{\left(#1\right.}}
\providecommand{\rbrak}[1]{\ensuremath{\left.#1\right)}}
\providecommand{\cbrak}[1]{\ensuremath{\left\{#1\right\}}}
\providecommand{\lcbrak}[1]{\ensuremath{\left\{#1\right.}}
\providecommand{\rcbrak}[1]{\ensuremath{\left.#1\right\}}}
\theoremstyle{remark}
\newtheorem{rem}{Remark}
\newcommand{\sgn}{\mathop{\mathrm{sgn}}}
\providecommand{\abs}[1]{\left\vert#1\right\vert}
\providecommand{\res}[1]{\Res\displaylimits_{#1}} 
\providecommand{\norm}[1]{\left\lVert#1\right\rVert}
%\providecommand{\norm}[1]{\lVert#1\rVert}
\providecommand{\mtx}[1]{\mathbf{#1}}
\providecommand{\mean}[1]{E\left[ #1 \right]}
\providecommand{\fourier}{\overset{\mathcal{F}}{ \rightleftharpoons}}
%\providecommand{\hilbert}{\overset{\mathcal{H}}{ \rightleftharpoons}}
\providecommand{\system}{\overset{\mathcal{H}}{ \longleftrightarrow}}
	%\newcommand{\solution}[2]{\textbf{Solution:}{#1}}
\newcommand{\solution}{\noindent \textbf{Solution: }}
\newcommand{\cosec}{\,\text{cosec}\,}
\providecommand{\dec}[2]{\ensuremath{\overset{#1}{\underset{#2}{\gtrless}}}}
\newcommand{\myvec}[1]{\ensuremath{\begin{pmatrix}#1\end{pmatrix}}}
\newcommand{\mydet}[1]{\ensuremath{\begin{vmatrix}#1\end{vmatrix}}}
\numberwithin{equation}{subsection}
\makeatletter
\@addtoreset{figure}{problem}
\makeatother
\let\StandardTheFigure\thefigure
\let\vec\mathbf
\renewcommand{\thefigure}{\theproblem}
\def\putbox#1#2#3{\makebox[0in][l]{\makebox[#1][l]{}\raisebox{\baselineskip}[0in][0in]{\raisebox{#2}[0in][0in]{#3}}}}
     \def\rightbox#1{\makebox[0in][r]{#1}}
     \def\centbox#1{\makebox[0in]{#1}}
     \def\topbox#1{\raisebox{-\baselineskip}[0in][0in]{#1}}
     \def\midbox#1{\raisebox{-0.5\baselineskip}[0in][0in]{#1}}
\vspace{3cm}
\title{Assignment 16}
\author{Pulkit Saxena}
\maketitle

\renewcommand{\thefigure}{\theenumi}
\renewcommand{\thetable}{\theenumi}

\section{\textbf{Problem Hoffman Pg 242 Q7}}
 Find the minimal polynomials and the rational forms of  the following real matrices\\
 \begin{enumerate}
     \item \myvec{0&-1&-1\\1&0&0\\-1&0&0}
     \item \myvec{c&0&-1\\0&c&1\\1&1&c}
     \item \myvec{\cos\theta&\sin\theta\\-\sin\theta&\cos\theta}
 \end{enumerate}

 
\section{\textbf{Theorems}}
\renewcommand{\thetable}{1}
\begin{longtable}{|l|l|}
\hline
\multirow{3}{*}{Theorem 1} & \\
&A Rational canonical form is a matrix $\vec{R}$ that is Direct sum of companion matrix.\\
&\\
&$\vec{R}=\vec{C(p_1)}\oplus\dots\oplus\vec{C(p_r)}$\\
&\\
&\parbox{10cm}
	{\begin{align}
	\vec{R}=
	\myvec{\vec{C(p_1)}&0&0&\dots&0\\
	0&\vec{C(p_2)}&0&\dots&0\\
	\vdots & \vdots & \vdots & \dots & \vdots\\
	0&0&0&\dots&\vec{C(p_r)}}
	\end{align}}\\
	&\\
&where $\vec{C(p_i)}$ is the $k_i$ x $k_i$ companion matrix of $p_i$ where polynomial $p_1,p_2\dots p_r$ are called\\
&\\
&invariant factors for Given Matrix .Where $k_i$ denotes the degree of annihilator of $p_i$.\\
&\\
&This representation is called rational form.\\

\hline
\multirow{3}{*}{Theorem 2} & \\
&If $p_i(x)=x+a_0$ then its companion matrix $\vec{C(p)}$ is 1 x 1 matrix as $\myvec{-a_0}$.\\
&\\
&If $k_i$ $\geq 2$ then $p(x)=x^k+a_{k-1}x^{k-1}+\dots+a_1x+a_0 $ then its companion matrix is\\
&\\
&\parbox{10cm}
	{\begin{align}
	\vec{C(p_i)}=
	\myvec{0&0&0&\dots&0&-a_0\\
	1&0&0&\dots&0&-a_1\\
	0&1&0&\dots&0&-a_2\\
	0&0&1&\dots&0&-a_3\\
	\vdots & \vdots & \vdots & \dots & \vdots& \vdots\\
	0&0&0&\dots&1&-a_{k-1}}
	\end{align}}\\

\hline
\caption{Illustration of theorem.}
\label{table:1}
\end{longtable}
\newpage
\section{\textbf{Solution}}
\renewcommand{\thetable}{2}
\begin{longtable}{|l|l|}
\hline
\multirow{3}{*}{Given part 1} & \\
&$\vec{A}=\myvec{0&-1&-1\\1&0&0\\-1&0&0}$\\
&\\
\hline


\multirow{3}{*}{Characteristics and Minimal Polynomial} & \\
&
Characteristics polynomial of the matrix  is $det(x\vec{I}-\vec{A})$\\ 
&\\
& $\det(x\vec{I}-\vec{A})$= $\left|
                \begin{array}{ccc}
                (x) & 1 & 1\\
                -1 & (x) & 0\\
                1 & 0 & (x)
                \end{array} \right|$ =$x\brak{x^2}-1\brak{-x}-x=x^3$\\
&\\
& Characteristic Polynomial = $x^3$\\
&\\
& Minimal Polynomial can be $x,x^2$ or $x^3$ of lowest degree\\
&\\
&satisfying $p(\vec{A})=0$    \\
&\\
&Let take $p\brak{x}=x\implies p(\vec{A})=\vec{A}=\myvec{0&-1&-1\\1&0&0\\-1&0&0}\neq0$\\
&\\
&Let take $p\brak{x}=x^2\implies p(\vec{A})=\vec{A}^2=\myvec{0&-1&-1\\1&0&0\\-1&0&0}\neq0$\\
&\\
&Take $p\brak{x}=x^3\implies p\brak{\vec{A}} =\vec{A}^3=\myvec{0&0&0\\0&0&0\\0&0&0}=\vec{0}$\\
&\\
&Thus minimal polynomial $p\brak{x}=x^3$.\\  
&\\
\hline
\multirow{3}{*}{Companion Matrix and Rational Form} & \\
&Since\\
&\\
&Characteristics polynomial$=$Minimal polynomial$=$Invariant factors\\
&\\
&$p\brak{x}=x^3+0x^2+0x+0$\\
&\\
&So Companion Matrix is of dimention 3x3 and from theorem 2\\
&\\
&$\vec{C(p)}=\myvec{0&0&0\\1&0&0\\0&1&0}$\\
&\\
&Since there is only one minimal polynomial of degree 3 
which is\\ &equal to characteristics equation therefore\\ &Rational matix=companion matrix\\ 
&$\vec{R}=\vec{C(p)}=\myvec{0&0&0\\1&0&0\\0&1&0}$\\
&\\
&which is in rational form.\\
&\\
\hline \hline
&\\




\multirow{3}{*}{Given Part 2} & \\
&$\vec{A}=\myvec{c&0&-1\\0&c&1\\1&1&c}$\\
&\\
\hline


\multirow{3}{*}{Characteristics and Minimal Polynomial} & \\
&
Characteristics polynomial of the matrix  is $det(x\vec{I}-\vec{A})$\\ 
&\\
& $\det(x\vec{I}-\vec{A})$= $\left|
                \begin{array}{ccc}
                (x-c) & 0 & 1\\
                0 & (x-c) & -1\\
                -1 & -1 & (x-c)
                \end{array} \right|$\\ &=$\brak{x-c}\brak{\brak{x-c}^2+1}-\brak{x-c}=\brak{x-c}^3$\\
&\\
& Characteristic Polynomial = $\brak{x-c}^3$\\
&\\
& Minimal Polynomial can be $\brak{x-c},\brak{x-c}^2$ or $\brak{x-c}^3$ \\
&\\
&of lowest degree satisfying $p(\vec{A})=0$    \\
&\\
&Let take $p\brak{x}=\brak{x-c}\implies p(\vec{A})=\vec{A}-c\vec{I}=\myvec{0&0&-1\\0&0&1\\1&1&0}\neq0$\\
&\\
&Let take $p\brak{x}=\brak{x-c}^2\implies p(\vec{A})=\brak{\vec{A}-c\vec{I}}^2=\myvec{-1&-1&0\\1&1&0\\0&0&0}\neq0$\\
&\\
&Take $p\brak{x}=\brak{x-c}^3\implies p\brak{\vec{A}} =\brak{\vec{A}-c\vec{I}}^3=\myvec{0&0&0\\0&0&0\\0&0&0}=\vec{0}$\\
&\\
&Thus minimal polynomial $p\brak{x}=\brak{x-c}^3$.\\  
&\\
\hline
\multirow{3}{*}{Companion Matrix and Rational Form} & \\
&Since\\
&\\
&Characteristics polynomial$=$Minimal polynomial$=$Invariant factors\\
&\\
&$p\brak{x}=x^3-3cx^2+3c^2x-c^3$\\
&\\
&So Companion Matrix is of dimention 3x3 and from theorem 2\\
&\\
&$\vec{C(p)}=\myvec{0&0&c^3\\1&0&-3c^2\\0&1&3c}$\\
&\\
&Since there is only one minimal polynomial of degree 3 
which is\\ &equal to characteristics equation therefore\\ &Rational matix=companion matrix\\ 
&$\vec{R}=\vec{C(p)}=\myvec{0&0&c^3\\1&0&-3c^2\\0&1&3c}$\\
&\\
&which is in rational form.\\
&\\
\hline \hline
\multirow{3}{*}{Given part 3} & \\
&$\vec{A}=\myvec{\cos\theta&\sin\theta\\-\sin\theta&\cos\theta}$\\
&\\
\hline


\multirow{3}{*}{Characteristics and Minimal Polynomial} & \\
&
Characteristics polynomial of the matrix  is $det(x\vec{I}-\vec{A})$\\ 
&\\
& $\det(x\vec{I}-\vec{A})$= $\left|
                \begin{array}{ccc}
                (x-\cos\theta) & -\sin\theta\\
                \sin\theta & (x-\cos\theta)\\
                \end{array} \right|$ =$x^2-2\cos\theta x +1$\\
&\\
& Characteristic Polynomial =Minimal Polynomial=$x^2-2\cos\theta x +1$\\
&\\

\hline
\multirow{3}{*}{Companion Matrix and Rational Form} & \\
&Since\\
&\\
&Characteristics polynomial$=$Minimal polynomial$=$Invariant factors\\
&\\
&$p\brak{x}=x^2-2\cos\theta x +1$\\
&\\
&So Companion Matrix is of dimention 2x2 and from theorem 2\\
&\\
&$\vec{C(p)}=\myvec{0&-1\\1&2\cos\theta}$\\
&\\
&Since there is only one minimal polynomial of degree 3 
which is\\ &equal to characteristics equation therefore\\ &Rational matix=companion matrix\\ 
&$\vec{R}=\vec{C(p)}=\vec{C(p)}=\myvec{0&-1\\1&2\cos\theta}$\\
&\\
&which is in rational form.\\
&\\
\hline

\caption{Solution Table}
\label{table:2}
\end{longtable}
\end{document}
