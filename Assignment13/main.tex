\documentclass[journal,12pt]{IEEEtran}
\usepackage{longtable}
\usepackage{setspace}
\usepackage{gensymb}
\singlespacing
\usepackage[cmex10]{amsmath}
\newcommand\myemptypage{
	\null
	\thispagestyle{empty}
	\addtocounter{page}{-1}
	\newpage
}
\usepackage{amsthm}
\usepackage{mdframed}
\usepackage{mathrsfs}
\usepackage{txfonts}
\usepackage{stfloats}
\usepackage{bm}
\usepackage{cite}
\usepackage{cases}
\usepackage{subfig}

\usepackage{longtable}
\usepackage{multirow}

\usepackage{enumitem}
\usepackage{mathtools}
\usepackage{steinmetz}
\usepackage{tikz}
\usepackage{circuitikz}
\usepackage{verbatim}
\usepackage{tfrupee}
\usepackage[breaklinks=true]{hyperref}
\usepackage{graphicx}
\usepackage{tkz-euclide}

\usetikzlibrary{calc,math}
\usepackage{listings}
    \usepackage{color}                                            %%
    \usepackage{array}                                            %%
    \usepackage{longtable}                                        %%
    \usepackage{calc}                                             %%
    \usepackage{multirow}                                         %%
    \usepackage{hhline}                                           %%
    \usepackage{ifthen}                                           %%
    \usepackage{lscape}     
\usepackage{multicol}
\usepackage{chngcntr}

\DeclareMathOperator*{\Res}{Res}

\renewcommand\thesection{\arabic{section}}
\renewcommand\thesubsection{\thesection.\arabic{subsection}}
\renewcommand\thesubsubsection{\thesubsection.\arabic{subsubsection}}

\renewcommand\thesectiondis{\arabic{section}}
\renewcommand\thesubsectiondis{\thesectiondis.\arabic{subsection}}
\renewcommand\thesubsubsectiondis{\thesubsectiondis.\arabic{subsubsection}}


\hyphenation{op-tical net-works semi-conduc-tor}
\def\inputGnumericTable{}                                 %%

\lstset{
%language=C,
frame=single, 
breaklines=true,
columns=fullflexible
}
\begin{document}
\onecolumn

\newtheorem{theorem}{Theorem}[section]
\newtheorem{problem}{Problem}
\newtheorem{proposition}{Proposition}[section]
\newtheorem{lemma}{Lemma}[section]
\newtheorem{corollary}[theorem]{Corollary}
\newtheorem{example}{Example}[section]
\newtheorem{definition}[problem]{Definition}

\newcommand{\BEQA}{\begin{eqnarray}}
\newcommand{\EEQA}{\end{eqnarray}}
\newcommand{\define}{\stackrel{\triangle}{=}}
\bibliographystyle{IEEEtran}
\raggedbottom
\setlength{\parindent}{0pt}
\providecommand{\mbf}{\mathbf}
\providecommand{\pr}[1]{\ensuremath{\Pr\left(#1\right)}}
\providecommand{\qfunc}[1]{\ensuremath{Q\left(#1\right)}}
\providecommand{\sbrak}[1]{\ensuremath{{}\left[#1\right]}}
\providecommand{\lsbrak}[1]{\ensuremath{{}\left[#1\right.}}
\providecommand{\rsbrak}[1]{\ensuremath{{}\left.#1\right]}}
\providecommand{\brak}[1]{\ensuremath{\left(#1\right)}}
\providecommand{\lbrak}[1]{\ensuremath{\left(#1\right.}}
\providecommand{\rbrak}[1]{\ensuremath{\left.#1\right)}}
\providecommand{\cbrak}[1]{\ensuremath{\left\{#1\right\}}}
\providecommand{\lcbrak}[1]{\ensuremath{\left\{#1\right.}}
\providecommand{\rcbrak}[1]{\ensuremath{\left.#1\right\}}}
\theoremstyle{remark}
\newtheorem{rem}{Remark}
\newcommand{\sgn}{\mathop{\mathrm{sgn}}}
\providecommand{\abs}[1]{\left\vert#1\right\vert}
\providecommand{\res}[1]{\Res\displaylimits_{#1}} 
\providecommand{\norm}[1]{\left\lVert#1\right\rVert}
%\providecommand{\norm}[1]{\lVert#1\rVert}
\providecommand{\mtx}[1]{\mathbf{#1}}
\providecommand{\mean}[1]{E\left[ #1 \right]}
\providecommand{\fourier}{\overset{\mathcal{F}}{ \rightleftharpoons}}
%\providecommand{\hilbert}{\overset{\mathcal{H}}{ \rightleftharpoons}}
\providecommand{\system}{\overset{\mathcal{H}}{ \longleftrightarrow}}
	%\newcommand{\solution}[2]{\textbf{Solution:}{#1}}
\newcommand{\solution}{\noindent \textbf{Solution: }}
\newcommand{\cosec}{\,\text{cosec}\,}
\providecommand{\dec}[2]{\ensuremath{\overset{#1}{\underset{#2}{\gtrless}}}}
\newcommand{\myvec}[1]{\ensuremath{\begin{pmatrix}#1\end{pmatrix}}}
\newcommand{\mydet}[1]{\ensuremath{\begin{vmatrix}#1\end{vmatrix}}}
\numberwithin{equation}{subsection}
\makeatletter
\@addtoreset{figure}{problem}
\makeatother
\let\StandardTheFigure\thefigure
\let\vec\mathbf
\renewcommand{\thefigure}{\theproblem}
\def\putbox#1#2#3{\makebox[0in][l]{\makebox[#1][l]{}\raisebox{\baselineskip}[0in][0in]{\raisebox{#2}[0in][0in]{#3}}}}
     \def\rightbox#1{\makebox[0in][r]{#1}}
     \def\centbox#1{\makebox[0in]{#1}}
     \def\topbox#1{\raisebox{-\baselineskip}[0in][0in]{#1}}
     \def\midbox#1{\raisebox{-0.5\baselineskip}[0in][0in]{#1}}
\vspace{3cm}
\title{Assignment 13}
\author{Pulkit Saxena}
\maketitle
\renewcommand{\thefigure}{\theenumi}
\renewcommand{\thetable}{\theenumi}
\section{\textbf{Problem hoffman pg208/1A}}
%
Find an invertible matrix $\vec{P}$ such that $\vec{P^{-1}AP}$ and $\vec{P^{-1}BP}$ are both diagonal where $\vec{A}$ and $\vec{B}$ are real matrices.
\begin{enumerate}
    \item $\Vec{A}=\myvec{1&2\\0&2} , \Vec{B}=\myvec{3&-8\\0&-1}$ 
    \end{enumerate}
\section{\textbf{Theorem's}}
\renewcommand{\thetable}{1}
\begin{table}[ht!]
\centering
\begin{tabular}{|c|l|}
   
	\hline
	\multirow{3}{*}{Theorem}
    &\\
    & According  to theorem , if a $2\times 2$ matrix has two characteristics \\
    &\\
    & values then the $\vec{P}$ that diagonalize $\vec{A}$ will
    necessarily also   \\
    &\\
    & diagonalize any $\vec{B}$ that commutes with $\vec{A}$. \\
    &\\
    \hline 
    &\\
    Common Basis
    &Let there exist a $\vec{P}$ in basis $\vec{\beta}=\{\vec{b}_1,.....,\vec{b}_n\}$ of $\mathbb{V}$ consisting of eigen vector\\& which are common to both $\vec{A}$ and $\vec{B}$ such that\\
    &\\
    &$\qquad\qquad\qquad\vec{A}\vec{b}_i=\lambda_{i}\vec{b}_i$\\
    &\\
    &$\qquad\qquad\qquad\vec{B}\vec{b}_i=\mu_{i}\vec{b}_i$\\
    &\\
    &$\qquad\qquad\qquad\Lambda_{A}=\myvec{\lambda_1&0\\0&\lambda_2}$\\
    &\\
    &$\qquad\qquad\qquad\Lambda_{B}=\myvec{\mu_1&0\\0&\mu_2}$\\
    &\\
    &$\qquad\qquad\qquad\Lambda_{A}=\vec{P^{-1}AP}$\\
    &\\
    &$\qquad\qquad\qquad\Lambda_{B}=\vec{P^{-1}BP}$\\
    &\\
    \hline

    
\end{tabular}
\label{table:1}
    \caption{Theorems }
\end{table}
\newpage
\section{\textbf{Solution}}
\renewcommand{\thetable}{2}
\begin{longtable}{|c|l|l|}
   \hline
   \multirow{3}{*}
   &&\\
   Operations&Matrix A&Matrix B\\
   \hline
   &&\\
   Characteristic Polynomial&$p\brak{x}=\mydet{x\Vec{I}-\Vec{A}}$&$p\brak{x}=\mydet{x\Vec{I}-\Vec{B}}$\\
	& $\qquad = \mydet{x-1&-2\\0&x-2}$ &$\qquad = \mydet{x-3&8\\0&x+1}$ \\
	&$\qquad=\brak{x-1}\brak{x-2}$&$\qquad=\brak{x-3}\brak{x+1}$ \\
 \hline
 &&\\
 Characteristic values&$p\brak{x}=0$&$p\brak{x}=0$\\
 &$\brak{x-1}\brak{x-2}=0$&$\brak{x-3}\brak{x+1}=0$\\
 &$\lambda_1=1$ ,$\lambda_2=2$&$\mu_1=3$ , $\mu_2=-1$\\ 
 \hline
 &&\\
 Basis for Characteristics Values& For $\lambda_1=1$& For $\mu_1=3$\\
 &&\\
 &$(\vec{A}-\lambda_1\vec{I})\vec{b_1}=0$&$(\vec{B}-\mu_1\vec{I})\vec{b_1}=0$\\
 &&\\
 &$\brak{\myvec{1&2\\0&2}-\myvec{1&0\\0&1}}\vec{b_1}=0$&$\brak{\myvec{3&-8\\0&-1}-3\myvec{1&0\\0&1}}\vec{b_1}=0$\\
 &&\\
 &$\brak{\myvec{0&2\\0&1}}\vec{b_1}=0$&$\brak{\myvec{0&-8\\0&-4}}\vec{b_1}=0$\\
 &&\\
 & $\vec{b_1}=\myvec{1\\0}$&$\vec{b_1}=\myvec{1\\0}$\\
 &&\\
 & For $\lambda_2=2$& For $\mu_2=-1$\\
 &&\\
 &$(\vec{A}-\lambda_2\vec{I})\vec{b_2}=0$&$(\vec{B}-\mu_2\vec{I})\vec{b_2}=0$\\
 &&\\
 &$\brak{\myvec{1&2\\0&2}-2\myvec{1&0\\0&1}}\vec{b_2}=0$&$\brak{\myvec{3&-8\\0&-1}-(-1)\myvec{1&0\\0&1}}\vec{b_2}=0$\\
 &&\\
 &$\brak{\myvec{-1&2\\0&0}}\vec{b_2}=0$&$\brak{\myvec{4&-8\\0&0}}\vec{b_2}=0$\\
 &&\\
 & $\vec{b_2}=\myvec{2\\1}$&$\vec{b_2}=\myvec{2\\1}$\\
 &&\\
 &$\vec{P}=\myvec{1&2\\0&1}$&$\vec{P}=\myvec{1&2\\0&1}$\\
 \hline
 \newpage
 \hline
 \multirow{3}{*}
 &&\\
 Verification& $\Lambda_A=\myvec{1&0\\0&2}$ & $\Lambda_B=\myvec{3&0\\0&-1}$\\
 &&\\
 &$\Lambda_{A}=\vec{P^{-1}AP}$&$\Lambda_{B}=\vec{P^{-1}BP}$\\
 &&\\
 &$\implies\myvec{1&-2\\0&1}\myvec{1&2\\0&2}\myvec{1&2\\0&1}$&$\implies\myvec{1&-2\\0&1}\myvec{3&-8\\0&-1}\myvec{1&2\\0&1}$\\
 &&\\
&$=\myvec{1&0\\0&2}=\Lambda_A$ & $=\myvec{3&0\\0&-1}=\Lambda_B$\\
&&\\
\hline
&&\\
Answer.The Invertible Matrix $\vec{P}$&$\vec{P}=\myvec{1&2\\0&1}$&$\vec{P}=\myvec{1&2\\0&1}$\\
&&\\
\hline
	
	\caption{Solution Table}
    \label{table:2}
\end{longtable}
\end{document}
