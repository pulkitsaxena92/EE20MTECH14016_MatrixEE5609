\documentclass[journal,12pt,twocolumn]{IEEEtran}

\usepackage{setspace}
\usepackage{gensymb}

\singlespacing


\usepackage[cmex10]{amsmath}

\usepackage{amsthm}

\usepackage{mathrsfs}
\usepackage{txfonts}
\usepackage{stfloats}
\usepackage{bm}
\usepackage{cite}
\usepackage{cases}
\usepackage{subfig}

\usepackage{longtable}
\usepackage{multirow}

\usepackage{enumitem}
\usepackage{mathtools}
\usepackage{steinmetz}
\usepackage{tikz}
\usepackage{circuitikz}
\usepackage{verbatim}
\usepackage{tfrupee}
\usepackage[breaklinks=true]{hyperref}
\usepackage{graphicx}
\usepackage{tkz-euclide}

\usetikzlibrary{calc,math}
\usepackage{listings}
    \usepackage{color}                                            %%
    \usepackage{array}                                            %%
    \usepackage{longtable}                                        %%
    \usepackage{calc}                                             %%
    \usepackage{multirow}                                         %%
    \usepackage{hhline}                                           %%
    \usepackage{ifthen}                                           %%
    \usepackage{lscape}     
\usepackage{multicol}
\usepackage{chngcntr}

\DeclareMathOperator*{\Res}{Res}

\renewcommand\thesection{\arabic{section}}
\renewcommand\thesubsection{\thesection.\arabic{subsection}}
\renewcommand\thesubsubsection{\thesubsection.\arabic{subsubsection}}

\renewcommand\thesectiondis{\arabic{section}}
\renewcommand\thesubsectiondis{\thesectiondis.\arabic{subsection}}
\renewcommand\thesubsubsectiondis{\thesubsectiondis.\arabic{subsubsection}}


\hyphenation{op-tical net-works semi-conduc-tor}
\def\inputGnumericTable{}                                 %%

\lstset{
%language=C,
frame=single, 
breaklines=true,
columns=fullflexible
}
\begin{document}


\newtheorem{theorem}{Theorem}[section]
\newtheorem{problem}{Problem}
\newtheorem{proposition}{Proposition}[section]
\newtheorem{lemma}{Lemma}[section]
\newtheorem{corollary}[theorem]{Corollary}
\newtheorem{example}{Example}[section]
\newtheorem{definition}[problem]{Definition}

\newcommand{\BEQA}{\begin{eqnarray}}
\newcommand{\EEQA}{\end{eqnarray}}
\newcommand{\define}{\stackrel{\triangle}{=}}
\bibliographystyle{IEEEtran}
\providecommand{\mbf}{\mathbf}
\providecommand{\pr}[1]{\ensuremath{\Pr\left(#1\right)}}
\providecommand{\qfunc}[1]{\ensuremath{Q\left(#1\right)}}
\providecommand{\sbrak}[1]{\ensuremath{{}\left[#1\right]}}
\providecommand{\lsbrak}[1]{\ensuremath{{}\left[#1\right.}}
\providecommand{\rsbrak}[1]{\ensuremath{{}\left.#1\right]}}
\providecommand{\brak}[1]{\ensuremath{\left(#1\right)}}
\providecommand{\lbrak}[1]{\ensuremath{\left(#1\right.}}
\providecommand{\rbrak}[1]{\ensuremath{\left.#1\right)}}
\providecommand{\cbrak}[1]{\ensuremath{\left\{#1\right\}}}
\providecommand{\lcbrak}[1]{\ensuremath{\left\{#1\right.}}
\providecommand{\rcbrak}[1]{\ensuremath{\left.#1\right\}}}
\theoremstyle{remark}
\newtheorem{rem}{Remark}
\newcommand{\sgn}{\mathop{\mathrm{sgn}}}
\providecommand{\abs}[1]{\left\vert#1\right\vert}
\providecommand{\res}[1]{\Res\displaylimits_{#1}} 
\providecommand{\norm}[1]{\left\lVert#1\right\rVert}
%\providecommand{\norm}[1]{\lVert#1\rVert}
\providecommand{\mtx}[1]{\mathbf{#1}}
\providecommand{\mean}[1]{E\left[ #1 \right]}
\providecommand{\fourier}{\overset{\mathcal{F}}{ \rightleftharpoons}}
%\providecommand{\hilbert}{\overset{\mathcal{H}}{ \rightleftharpoons}}
\providecommand{\system}{\overset{\mathcal{H}}{ \longleftrightarrow}}
	%\newcommand{\solution}[2]{\textbf{Solution:}{#1}}
\newcommand{\solution}{\noindent \textbf{Solution: }}
\newcommand{\cosec}{\,\text{cosec}\,}
\providecommand{\dec}[2]{\ensuremath{\overset{#1}{\underset{#2}{\gtrless}}}}
\newcommand{\myvec}[1]{\ensuremath{\begin{pmatrix}#1\end{pmatrix}}}
\newcommand{\mydet}[1]{\ensuremath{\begin{vmatrix}#1\end{vmatrix}}}
\numberwithin{equation}{subsection}
\makeatletter
\@addtoreset{figure}{problem}
\makeatother
\let\StandardTheFigure\thefigure
\let\vec\mathbf
\renewcommand{\thefigure}{\theproblem}
\def\putbox#1#2#3{\makebox[0in][l]{\makebox[#1][l]{}\raisebox{\baselineskip}[0in][0in]{\raisebox{#2}[0in][0in]{#3}}}}
     \def\rightbox#1{\makebox[0in][r]{#1}}
     \def\centbox#1{\makebox[0in]{#1}}
     \def\topbox#1{\raisebox{-\baselineskip}[0in][0in]{#1}}
     \def\midbox#1{\raisebox{-0.5\baselineskip}[0in][0in]{#1}}
\vspace{3cm}
\title{Assignment 7}
\author{Pulkit Saxena}
\maketitle
\newpage
\bigskip
\renewcommand{\thefigure}{\theenumi}
\renewcommand{\thetable}{\theenumi}
\section{\textbf{Problem}}
1)Find QR decomposition of matrix
\begin{equation}
	\vec{V} = \myvec{9 & -12\\ -12 & 16}
\end{equation}
2)Find the vertex $\vec{c}$ of the parabola using SVD for
\begin{align}
   9x^2-24xy+16y^2-18x-101y+19 = 0 \label{eq:conic}
\end{align}
by changing  $\eta$ to $\eta/2$ also verify the result using least squares.
\section{\textbf{Solution}}
\subsection{Part 1:QR Decomposition of V}
Let $\vec{x}$ and $\vec{y}$ be the column vectors of the given matrix.
\begin{align}
    \vec{x} &= \myvec{9 \\-12 }\\
    \vec{y} &= \myvec{-12 \\ 16}
\end{align}
The column vectors can be expressed as follows,
\begin{align}
    \vec{x} &= k_1\vec{u}_1\label{eq_QR1}\\
    \vec{y} &= r_1\vec{u}_1+k_2\vec{u}_2\label{eq_QR2}
\end{align}
\begin{align}
    k_1 &= \norm{\vec{x}}\label{eq1}\\
    \vec{u}_1 &= \frac{\vec{x}}{k_1}\\
    r_1 &= \frac{\vec{u}_1^T\vec{y}}{\norm{\vec{u}_1}^2}\\
    \vec{u}_2 &= \frac{\vec{y} - r_1 \vec{u}_1}{\norm{\vec{y} - r_1 \vec{u}_1}}\\
    k_2 &= {\vec{u}_2^T\vec{y}}\label{eq2}
\end{align}
The \eqref{eq_QR1} and \eqref{eq_QR2} can be written as, 
\begin{align}
\myvec{\vec{x} & \vec{y}} &= \myvec{\vec{u}_1 & \vec{u}_2}\myvec{k_1 & r_1 \\ 0 & k_2}\label{QRMain}\\
\myvec{\vec{x} & \vec{y}} &= \vec{Q}\vec{R}
\end{align}
Now, $\vec{R}$ is an upper triangular matrix and also,
\begin{align}
\vec{Q}^T\vec{Q}=\vec{I}
\end{align}
Now using equations \eqref{eq1} to \eqref{eq2} we get, 
\begin{align}
    k_1 &= \sqrt{9^2+12^2} = 15\label{eqval1}\\ 
    \vec{u}_1 &= \myvec{\frac{3}{5}\\ \\ \frac{-4}{5}} \\
    r_1 &= \myvec{\frac{3}{5}&&\frac{-4}{5}}\myvec{-12 \\ 16} = -20\\ 
    \vec{u}_2 &= \myvec{0 \\\\ 0} \\
    k_2 &= \myvec{0&&0}\myvec{-12\\16} = 0\label{eqval2} 
\end{align}
Thus putting the values from \eqref{eqval1} to \eqref{eqval2} in \eqref{QRMain} we obtain QR decomposition,
\begin{align}
    \myvec{9 & -12\\ -12 & 16} =\myvec{\frac{3}{5}&&0\\\\\frac{-4}{5}&&0}\myvec{15 && -20\\\\0&&0}
\end{align}
\subsection{Part 2:Finding Vertex using SVD}
\begin{align}
\label{eq V}\vec{V} = \myvec{9&-12\\-12&16}\\ 
\label{eq U} \vec{u} = \myvec{-9\\-\frac{101}{2}}\\ 
f = 19 \label{eq f}
\end{align}
\begin{align}
\vec{P}&=\myvec{\vec{p_1}&\vec{p_2}}=\frac{1}{5}\myvec{-4&-3\\ -3 &4}\label{eq P}    
\end{align}
\begin{align}
\eta=\vec{u}^T\vec{p_1}=\frac{75}{2}
\end{align}
So the equation of perpendicular line passing through focus and intersecting parabola at vertex c is given as
\begin{align}
\myvec{\vec{u^T}+\frac{\eta}{2}\vec{p_1^T} \\ \vec{V}}\vec{c}=
\myvec{-f \\ \frac{\eta}{2}\vec{p_1}-\vec{u}} 
\end{align}

using \eqref{eq V},\eqref{eq U} ,\eqref{eq f} and \eqref{eq P}
\begin{align}
    \myvec{-24& \frac{-247}{4}\\ 9 & -12 \\  -12 & 16 }\vec{c}=\myvec{-19 \\ -6\\ \frac{157}{4}} 
\end{align}
\begin{align}
\vec{Mc=b} \label{1}
\end{align}
where
\begin{align}
\vec{M} = \myvec{-24 & \frac{-247}{4} \\ 9 & -12 \\ -12 & 16},b = \myvec{-19 \\-6\\\frac{157}{4}}\label{2}	
\end{align}
To solve \eqref{1}, we perform singular value decomposition on $\vec{M}$ given as 
\begin{align}
	\vec{M = USV^T }\label{3}
\end{align}
Substituting the value of $\vec{M}$ from \eqref{3} in \eqref{1}, we get
\begin{align}
	&\vec{USV^T}\vec{c} = \vec{b} \\
\implies& \vec{c} = \vec{VS_+U^T}\vec{b}\label{4}
\end{align}
where, $\vec{S_+}$ is Moore-Pen-rose Pseudo-Inverse of $\vec{S}$. Columns of $\vec{U}$ are eigen-vectors of $\vec{MM^T}$, columns of $\vec{V}$ are eigenvectors of $\vec{M^TM}$ and $\vec{S}$ is diagonal matrix of singular value of eigenvalues of $\vec{M^TM}$.
\begin{align}
\vec{M}^T\vec{M}=\myvec{801&1182\\1182&\frac{67409}{16}}\label{eqMTM}\\
\vec{M}\vec{M}^T=\myvec{\frac{70225}{16}&525&-700\\525&225&-300\\-700&-300&400}
\end{align}
Eigen values of $\vec{M^TM}$ can be found out as
\begin{align}
	 \abs{\vec{M^TM-\lambda I}} = 0
\end{align}
\begin{align}
\myvec{801-\lambda&1182\\1182&\frac{67409}{16}-\lambda}=0\label{eqMTM}
\end{align}
Hence eigen values of $\vec{M}^T\vec{M}$ are,
\begin{align}
\lambda_1 &=431.539\\
\lambda_2 &= 4582.523\\
\end{align}
Hence the eigen vectors of $\vec{M}^T\vec{M}$ are,
\begin{align}
\vec{v}_1=\myvec{-3.2\\1},
\vec{v}_2=\myvec{.312\\1}
\intertext{Normalizing the eigen vectors we get,}
\vec{v}_1=\myvec{-.9544\\.2982}
\vec{v}_2=\myvec{.2978\\.9546}
\end{align}
Hence we obtain $\vec{V}$ of \eqref{3} as follows,
\begin{align}
\vec{V}=\myvec{-.9544&.2978\\.2982&.9546}
\end{align}
Similarly,eigen values of $\vec{M}\vec{M}^T$ are,
\begin{align}
	 \abs{\vec{MM^T-\lambda I}} = 0\\
	 \myvec{\frac{70225}{16}-\lambda&525&-700\\525&225-\lambda&-300\\-700&-300&400-\lambda}=0
\end{align}
\begin{align}
\lambda_3 &=431.539\\
\lambda_4 &= 4582.531\\
\lambda_5 &=0
\end{align}
Hence the corresponding eigen vectors of $\vec{M}\vec{M}^T$ are,
\begin{align}
\vec{u}_1=\myvec{.27637\\-.75\\1},
\vec{u}_2=\myvec{-5.6536\\-.75\\1},
\vec{u}_3=\myvec{0\\1.33\\1}
\intertext{Normalizing the eigen vectors we get,}
\vec{u}_1=\myvec{.2159\\-.5859\\.78125},
\vec{u}_2=\myvec{-.9764\\-.1295\\.1727},
\vec{u}_3=\myvec{0\\.8\\.6}
\end{align}
Hence we obtain $\vec{U}$ of \eqref{3} as follows,
\begin{align}
\vec{U}=\myvec{.2159&-.9764&0\\-.5859&-.1295&.8\\.78125&.1727&.6}
\end{align}
After computing the singular values from eigen values $\lambda_3, \lambda_4, \lambda_5$ we get $\vec{S}$ of \eqref{3} as follows,
\begin{align}\label{eqS}
\vec{S}=\myvec{20.7735&0\\0&67.6943\\0&0}
\end{align}
From \eqref{3} we get the Singular Value Decomposition of $\vec{M}$ ,
\begin{align}
\vec{M} = \myvec{.2159&-.9764&0\\-.5859&-.1295&.8\\.78125&.1727&.6}\myvec{20.7735&0\\0&67.6943\\0&0}\\\myvec{-.9544&.2978\\.2982&.9546}^T
\end{align}
\begin{align}
=\myvec{-24 & 61.75 \\ 9 & -12 \\ -12 & 16}    
\end{align}
Moore-Penrose Pseudo inverse of $\vec{S}$ is given by,
\begin{align}
\vec{S_+} = \myvec{.0481&0&0\\0&.01477&0}
\end{align}
From \eqref{4} we get,
\begin{align}
\vec{U}^T\vec{b}&=\myvec{30.0754\\26.1070\\18.75}\\
\vec{S_+}\vec{U}^T\vec{b}&=\myvec{1.4477\\.38566}\\
\vec{c} = \vec{V}\vec{S_+}\vec{U}^T\vec{b} &= \myvec{-1.266\\.8}\\
\implies\vec{c}=\myvec{-1.266\\.8}\label{eq85}
\end{align}
Verifying the solution of \eqref{eq85} using,
\begin{align}
\vec{M}^T\vec{M}\vec{c} = \vec{M}^T\vec{b}\label{eqVerify}
\end{align}
Evaluating the R.H.S in \eqref{eqVerify} we get,
\begin{align}
\vec{M}^T\vec{M}\vec{c} &= \myvec{-69\\\frac{7493}{4}}\\
\implies\myvec{801&1182\\1182&\frac{67409}{16}}\vec{c} &= \myvec{-69\\\frac{7493}{4}}\label{eq:eq17}
\end{align}
Solving the augmented matrix of \eqref{eq:eq17} we get,
\begin{align}
\myvec{801&1182&-69\\1182&\frac{67409}{16}&\frac{7493}{4}} &\xleftrightarrow{R_1=\frac{1}{801}R_1}\myvec{1&\frac{394}{267}&\frac{-23}{267}\\1182&\frac{67409}{16}&\frac{7493}{4}}\\
&\xleftrightarrow{R_2=R_2-1182R_1}\myvec{1&\frac{394}{267}&\frac{-23}{267}\\0&\frac{3515625}{1424}&\frac{703125}{356}}\\
&\xleftrightarrow{R_2=\frac{1424}{3515625}R_2}\myvec{1&\frac{394}{267}&\frac{-23}{267}\\0&1&\frac{4}{5}}\\
&\xleftrightarrow{R_1=R_1-\frac{394}{267}R_2}\myvec{1&0&\frac{-19}{15}\\0&1&\frac{4}{5}}\label{eq:eq13}
\end{align}
From equation \eqref{eq:eq13}, solution is given by,
\begin{align}
\vec{c}=\myvec{\frac{-19}{15}\\\frac{4}{5}}\\
\vec{c}=\myvec{-1.266\\.8}\label{eq:eq14}
\end{align}
Comparing results of $\vec{c}$ from \eqref{eq85} and \eqref{eq:eq14}, we can say that the solution is verified.

\end{document}

