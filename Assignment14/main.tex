\documentclass[journal,12pt]{IEEEtran}
\usepackage{longtable}
\usepackage{setspace}
\usepackage{gensymb}
\singlespacing
\usepackage[cmex10]{amsmath}
\newcommand\myemptypage{
	\null
	\thispagestyle{empty}
	\addtocounter{page}{-1}
	\newpage
}
\usepackage{amsthm}
\usepackage{mdframed}
\usepackage{mathrsfs}
\usepackage{txfonts}
\usepackage{stfloats}
\usepackage{bm}
\usepackage{cite}
\usepackage{cases}
\usepackage{subfig}

\usepackage{longtable}
\usepackage{multirow}

\usepackage{enumitem}
\usepackage{mathtools}
\usepackage{steinmetz}
\usepackage{tikz}
\usepackage{circuitikz}
\usepackage{verbatim}
\usepackage{tfrupee}
\usepackage[breaklinks=true]{hyperref}
\usepackage{graphicx}
\usepackage{tkz-euclide}

\usetikzlibrary{calc,math}
\usepackage{listings}
    \usepackage{color}                                            %%
    \usepackage{array}                                            %%
    \usepackage{longtable}                                        %%
    \usepackage{calc}                                             %%
    \usepackage{multirow}                                         %%
    \usepackage{hhline}                                           %%
    \usepackage{ifthen}                                           %%
    \usepackage{lscape}     
\usepackage{multicol}
\usepackage{chngcntr}

\DeclareMathOperator*{\Res}{Res}

\renewcommand\thesection{\arabic{section}}
\renewcommand\thesubsection{\thesection.\arabic{subsection}}
\renewcommand\thesubsubsection{\thesubsection.\arabic{subsubsection}}

\renewcommand\thesectiondis{\arabic{section}}
\renewcommand\thesubsectiondis{\thesectiondis.\arabic{subsection}}
\renewcommand\thesubsubsectiondis{\thesubsectiondis.\arabic{subsubsection}}


\hyphenation{op-tical net-works semi-conduc-tor}
\def\inputGnumericTable{}                                 %%

\lstset{
%language=C,
frame=single, 
breaklines=true,
columns=fullflexible
}
\begin{document}
\onecolumn

\newtheorem{theorem}{Theorem}[section]
\newtheorem{problem}{Problem}
\newtheorem{proposition}{Proposition}[section]
\newtheorem{lemma}{Lemma}[section]
\newtheorem{corollary}[theorem]{Corollary}
\newtheorem{example}{Example}[section]
\newtheorem{definition}[problem]{Definition}

\newcommand{\BEQA}{\begin{eqnarray}}
\newcommand{\EEQA}{\end{eqnarray}}
\newcommand{\define}{\stackrel{\triangle}{=}}
\bibliographystyle{IEEEtran}
\raggedbottom
\setlength{\parindent}{0pt}
\providecommand{\mbf}{\mathbf}
\providecommand{\pr}[1]{\ensuremath{\Pr\left(#1\right)}}
\providecommand{\qfunc}[1]{\ensuremath{Q\left(#1\right)}}
\providecommand{\sbrak}[1]{\ensuremath{{}\left[#1\right]}}
\providecommand{\lsbrak}[1]{\ensuremath{{}\left[#1\right.}}
\providecommand{\rsbrak}[1]{\ensuremath{{}\left.#1\right]}}
\providecommand{\brak}[1]{\ensuremath{\left(#1\right)}}
\providecommand{\lbrak}[1]{\ensuremath{\left(#1\right.}}
\providecommand{\rbrak}[1]{\ensuremath{\left.#1\right)}}
\providecommand{\cbrak}[1]{\ensuremath{\left\{#1\right\}}}
\providecommand{\lcbrak}[1]{\ensuremath{\left\{#1\right.}}
\providecommand{\rcbrak}[1]{\ensuremath{\left.#1\right\}}}
\theoremstyle{remark}
\newtheorem{rem}{Remark}
\newcommand{\sgn}{\mathop{\mathrm{sgn}}}
\providecommand{\abs}[1]{\left\vert#1\right\vert}
\providecommand{\res}[1]{\Res\displaylimits_{#1}} 
\providecommand{\norm}[1]{\left\lVert#1\right\rVert}
%\providecommand{\norm}[1]{\lVert#1\rVert}
\providecommand{\mtx}[1]{\mathbf{#1}}
\providecommand{\mean}[1]{E\left[ #1 \right]}
\providecommand{\fourier}{\overset{\mathcal{F}}{ \rightleftharpoons}}
%\providecommand{\hilbert}{\overset{\mathcal{H}}{ \rightleftharpoons}}
\providecommand{\system}{\overset{\mathcal{H}}{ \longleftrightarrow}}
	%\newcommand{\solution}[2]{\textbf{Solution:}{#1}}
\newcommand{\solution}{\noindent \textbf{Solution: }}
\newcommand{\cosec}{\,\text{cosec}\,}
\providecommand{\dec}[2]{\ensuremath{\overset{#1}{\underset{#2}{\gtrless}}}}
\newcommand{\myvec}[1]{\ensuremath{\begin{pmatrix}#1\end{pmatrix}}}
\newcommand{\mydet}[1]{\ensuremath{\begin{vmatrix}#1\end{vmatrix}}}
\numberwithin{equation}{subsection}
\makeatletter
\@addtoreset{figure}{problem}
\makeatother
\let\StandardTheFigure\thefigure
\let\vec\mathbf
\renewcommand{\thefigure}{\theproblem}
\def\putbox#1#2#3{\makebox[0in][l]{\makebox[#1][l]{}\raisebox{\baselineskip}[0in][0in]{\raisebox{#2}[0in][0in]{#3}}}}
     \def\rightbox#1{\makebox[0in][r]{#1}}
     \def\centbox#1{\makebox[0in]{#1}}
     \def\topbox#1{\raisebox{-\baselineskip}[0in][0in]{#1}}
     \def\midbox#1{\raisebox{-0.5\baselineskip}[0in][0in]{#1}}
\vspace{3cm}
\title{Assignment 14}
\author{Pulkit Saxena}
\maketitle
\renewcommand{\thefigure}{\theenumi}
\renewcommand{\thetable}{\theenumi}
\section{\textbf{Problem Hoffman pg213 Q1}}
Let $\mathbb{V}$ be a finite-dimensional vector space and let $\mathbb{W}_1$ be any subspace of $\mathbb{V}$.Prove that there is a subspace $\mathbb{W}_2$ of $\mathbb{V}$ such that $\mathbb{V}$= $\mathbb{W}_1$ $\oplus$ $\mathbb{W}_2$
\section{\textbf{Solution}}
\renewcommand{\thetable}{2}
\begin{longtable}{|l|l|}
\hline
\multirow{3}{*}{Assumption and Claim} & \\
&Let $\beta=\{\vec{u}_1,.....,\vec{u}_n\}$ be a basis for $\mathbb{W}_1$.Since $\mathbb{W}_1$ is the subspace of $\mathbb{V}$.\\
&\\
& therefore let us take  $\alpha=\{\vec{u}_1,.....,\vec{u}_n,\vec{u}_{n+1},....,\vec{u}_m\}$ the basis of $\mathbb{V}$\\
&\\
&So $\mathbb{W}_2$=span\brak{\{\vec{u}_{n+1},......,\vec{u}_m\}}\\
&\\
&Claim that $\mathbb{V}=\mathbb{W}_1\oplus\mathbb{W}_2$.\\
&\\

\hline

\multirow{3}{*}{Proof of $\mathbb{V}=\mathbb{W}_1 + \mathbb{W}_2$} & \\
&if $\vec{v}\in \mathbb{V}$, then  \\
&\\
& $\vec{v}=\sum_{i=1}^{m}a_i\vec{u}_i=\sum_{i=1}^{n}a_i\vec{u}_i + \sum_{i=n+1}^{m}a_i\vec{u}_i\in \mathbb{W}_1+\mathbb{W}_2$ for scalar $a_i$ , $i=1,....,m$ \\
&\\
&This implies that $\mathbb{V}\subseteq\mathbb{W}_1 +\mathbb{W}_2$ But by the defination  of $\mathbb{W}_1 + \mathbb{W}_2$ \\
&\\
&we know that $\mathbb{W}_1 + \mathbb{W}_2 \subseteq\mathbb{V}$.\\
&\\
&Hence $\mathbb{V}=\mathbb{W}_1+\mathbb{W}_2$\\
&\\
\hline
\multirow{3}{*}{Proof of $\mathbb{W}_1\cap\mathbb{W}_2=\{0\}$} & \\
&Let $\vec{u}\in\mathbb{W}_1\cap\mathbb{W}_2$\\ 
&\\
&Then $\vec{u}=\sum_{i=1}^{n}b_i\vec{u}_i=\sum_{i=n+1}^{m}c_i\vec{u}_i$ for some scalar $b_1,....b_n,c_{n+1},.....,c_m$\\
&\\
&$\implies \sum_{i=1}^{n}b_i\vec{u}_i + \sum_{i=n+1}^{m}\brak{-c_i}\vec{u}_i =0$\\
&\\
&But $\alpha$ is linearly independent ,since $\alpha$ is a basis. \\
&\\
&Hence $b_1=.....=b_n=c_{n+1}.....=c_m=0$.This implies $\vec{u}=0$.\\ 
&\\
& Thus $\mathbb{W}_1\cap\mathbb{W}_2=\{0\}$\\
&\\

\hline
\multirow{3}{*}{Combining Both the proof} & \\
&$\mathbb{V}=\mathbb{W}_1+\mathbb{W}_2$\\
&\\
&$\mathbb{W}_1\cap\mathbb{W}_2=\{0\}$\\

\hline
&\\

&From  the above two condition we can say that $\mathbb{V}$ is the direct sum of \\
&\\
&subspaces $\mathbb{W}_1$ and $\mathbb{W}_2$ . Hence it is represented as\\
&\\
&$\mathbb{V}=\mathbb{W}_1\oplus\mathbb{W}_2$\\
&\\
&Hence Proved.\\
\hline
\caption{Solution Table}
\label{table:2}
\end{longtable}
\end{document}
