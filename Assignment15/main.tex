\documentclass[journal,12pt]{IEEEtran}
\usepackage{longtable}
\usepackage{setspace}
\usepackage{gensymb}
\singlespacing
\usepackage[cmex10]{amsmath}
\newcommand\myemptypage{
	\null
	\thispagestyle{empty}
	\addtocounter{page}{-1}
	\newpage
}
\usepackage{amsthm}
\usepackage{mdframed}
\usepackage{mathrsfs}
\usepackage{txfonts}
\usepackage{stfloats}
\usepackage{bm}
\usepackage{cite}
\usepackage{cases}
\usepackage{subfig}

\usepackage{longtable}
\usepackage{multirow}

\usepackage{enumitem}
\usepackage{mathtools}
\usepackage{steinmetz}
\usepackage{tikz}
\usepackage{circuitikz}
\usepackage{verbatim}
\usepackage{tfrupee}
\usepackage[breaklinks=true]{hyperref}
\usepackage{graphicx}
\usepackage{tkz-euclide}

\usetikzlibrary{calc,math}
\usepackage{listings}
    \usepackage{color}                                            %%
    \usepackage{array}                                            %%
    \usepackage{longtable}                                        %%
    \usepackage{calc}                                             %%
    \usepackage{multirow}                                         %%
    \usepackage{hhline}                                           %%
    \usepackage{ifthen}                                           %%
    \usepackage{lscape}     
\usepackage{multicol}
\usepackage{chngcntr}

\DeclareMathOperator*{\Res}{Res}

\renewcommand\thesection{\arabic{section}}
\renewcommand\thesubsection{\thesection.\arabic{subsection}}
\renewcommand\thesubsubsection{\thesubsection.\arabic{subsubsection}}

\renewcommand\thesectiondis{\arabic{section}}
\renewcommand\thesubsectiondis{\thesectiondis.\arabic{subsection}}
\renewcommand\thesubsubsectiondis{\thesubsectiondis.\arabic{subsubsection}}


\hyphenation{op-tical net-works semi-conduc-tor}
\def\inputGnumericTable{}                                 %%

\lstset{
%language=C,
frame=single, 
breaklines=true,
columns=fullflexible
}
\begin{document}
\onecolumn

\newtheorem{theorem}{Theorem}[section]
\newtheorem{problem}{Problem}
\newtheorem{proposition}{Proposition}[section]
\newtheorem{lemma}{Lemma}[section]
\newtheorem{corollary}[theorem]{Corollary}
\newtheorem{example}{Example}[section]
\newtheorem{definition}[problem]{Definition}

\newcommand{\BEQA}{\begin{eqnarray}}
\newcommand{\EEQA}{\end{eqnarray}}
\newcommand{\define}{\stackrel{\triangle}{=}}
\bibliographystyle{IEEEtran}
\raggedbottom
\setlength{\parindent}{0pt}
\providecommand{\mbf}{\mathbf}
\providecommand{\pr}[1]{\ensuremath{\Pr\left(#1\right)}}
\providecommand{\qfunc}[1]{\ensuremath{Q\left(#1\right)}}
\providecommand{\sbrak}[1]{\ensuremath{{}\left[#1\right]}}
\providecommand{\lsbrak}[1]{\ensuremath{{}\left[#1\right.}}
\providecommand{\rsbrak}[1]{\ensuremath{{}\left.#1\right]}}
\providecommand{\brak}[1]{\ensuremath{\left(#1\right)}}
\providecommand{\lbrak}[1]{\ensuremath{\left(#1\right.}}
\providecommand{\rbrak}[1]{\ensuremath{\left.#1\right)}}
\providecommand{\cbrak}[1]{\ensuremath{\left\{#1\right\}}}
\providecommand{\lcbrak}[1]{\ensuremath{\left\{#1\right.}}
\providecommand{\rcbrak}[1]{\ensuremath{\left.#1\right\}}}
\theoremstyle{remark}
\newtheorem{rem}{Remark}
\newcommand{\sgn}{\mathop{\mathrm{sgn}}}
\providecommand{\abs}[1]{\left\vert#1\right\vert}
\providecommand{\res}[1]{\Res\displaylimits_{#1}} 
\providecommand{\norm}[1]{\left\lVert#1\right\rVert}
%\providecommand{\norm}[1]{\lVert#1\rVert}
\providecommand{\mtx}[1]{\mathbf{#1}}
\providecommand{\mean}[1]{E\left[ #1 \right]}
\providecommand{\fourier}{\overset{\mathcal{F}}{ \rightleftharpoons}}
%\providecommand{\hilbert}{\overset{\mathcal{H}}{ \rightleftharpoons}}
\providecommand{\system}{\overset{\mathcal{H}}{ \longleftrightarrow}}
	%\newcommand{\solution}[2]{\textbf{Solution:}{#1}}
\newcommand{\solution}{\noindent \textbf{Solution: }}
\newcommand{\cosec}{\,\text{cosec}\,}
\providecommand{\dec}[2]{\ensuremath{\overset{#1}{\underset{#2}{\gtrless}}}}
\newcommand{\myvec}[1]{\ensuremath{\begin{pmatrix}#1\end{pmatrix}}}
\newcommand{\mydet}[1]{\ensuremath{\begin{vmatrix}#1\end{vmatrix}}}
\numberwithin{equation}{subsection}
\makeatletter
\@addtoreset{figure}{problem}
\makeatother
\let\StandardTheFigure\thefigure
\let\vec\mathbf
\renewcommand{\thefigure}{\theproblem}
\def\putbox#1#2#3{\makebox[0in][l]{\makebox[#1][l]{}\raisebox{\baselineskip}[0in][0in]{\raisebox{#2}[0in][0in]{#3}}}}
     \def\rightbox#1{\makebox[0in][r]{#1}}
     \def\centbox#1{\makebox[0in]{#1}}
     \def\topbox#1{\raisebox{-\baselineskip}[0in][0in]{#1}}
     \def\midbox#1{\raisebox{-0.5\baselineskip}[0in][0in]{#1}}
\vspace{3cm}
\title{Assignment 15}
\author{Pulkit Saxena}
\maketitle

\renewcommand{\thefigure}{\theenumi}
\renewcommand{\thetable}{\theenumi}

\section{\textbf{Problem Hoffman Pg 230 Q2}}
 Let $T$ be a linear operator on $\mathbb{R}^3$ which is represented in standard ordered basis by matrix\\
\begin{align}
    \myvec{2&0&0\\0&2&0\\0&0&-1}
\end{align}
Prove that $T$ has no cyclic vector.What is the $T$-cyclic subspace generated by the vector $\myvec{1\\-1\\3}$?\\ 
\section{\textbf{Theorems}}
\renewcommand{\thetable}{1}
\begin{longtable}{|l|l|}
\hline
\multirow{3}{*}{Theorem 1} & \\
&$T$ be a linear operator on vector space $\mathbb{V}$ of n dimensional.\\
&\\
&There exist a cyclic vector for $T$ if and only if minimal polynomial\\
&\\
&and characteristic polynomial are same.\\
&\\
&Characteristics Polynomial:-\\
&\\
&$f(x)=(\vec{x}-\lambda_1)^{d_1}............(\vec{x}-\lambda_k)^{d_k}$ \\
&\\
&Minimal Polynomial:-\\
&\\
&$p_a(x)=(\vec{x}-\lambda_1)............(\vec{x}-\lambda_k)$ for the given eigen values $\lambda_1.....\lambda_k$\\
&\\
\hline
\multirow{3}{*}{Theorem 2} & \\
&$\mathbb{Z}(\vec{a};T) $is the subspace spanned by vectors $T^k\vec{a}$ , $k\geq 0$ , and $\vec{a}$ is a cyclic vector for $T$\\
&\\
&if and only if these vector span $\mathbb{V}$,the $\alpha$ is called cyclic vector of $T$.\\ 
&\\
\hline
\multirow{3}{*}{Theorem 3} & \\
&Let $\vec{a}$ be any non-zero vector in $\mathbb{V}$ and let $p_a(minimal polynomial)$ be the $T$-annihilator of $\vec{a}$\\
Cyclic Base
&\\
&If the degree of $p_a$ is k , then vectors $\vec{a},T\vec{a},T^2\vec{a},.......,T^{k-1}\vec{a}$ form of a basis  for $\mathbb{Z}\brak{\vec{a};T}$ \\
&\\
&if $g\brak{T}\vec{a}=0$.\\
&\\
\hline
\caption{Illustration of theorem.}
\label{table:1}
\end{longtable}
\newpage
\section{\textbf{Solution}}
\renewcommand{\thetable}{2}
\begin{longtable}{|l|l|}
\hline

\multirow{3}{*}{proof of $T$ doesn't have cyclic vector } & \\
&
Characteristics polynomial of the matrix  is $det(x\vec{I}-\vec{A})$\\ 
&\\
& $\det(x\vec{I}-\vec{A})$= $\left|
                \begin{array}{ccc}
                (x-2) & 0 & 0\\
                0 & (x-2) & 0\\
                0 & 0 & (x+1)
                \end{array} \right|$  \\
&\\
& Characteristic Polynomial = $(x-2)^2(x+1)$\\
&\\
& Minimal Polynomial=$p_a(x)$=$(x-2)(x+1)$  degree =2   \\
&\\
&Minimal Polynomial $\neq$ Characteristic Polynomial \\
&\\
&Thus from Theorem 1 $T$ doesn't have cyclic vector.\\  
&\\
\hline
\multirow{3}{*}{Cyclic subspace} & \\

&For the given matrix  $\myvec{2&0&0\\0&2&0\\0&0&-1}$\\
&\\
&$T\brak{\vec{x},\vec{y},\vec{z}}=\brak{2\vec{x},2\vec{y},-\vec{z}}$\\
&\\
&$T\myvec{1\\-1\\3}=\myvec{2\\-2\\-3}$\\
&\\
&Since we know $T^2\vec{a}=T\brak{T\vec{a}}$\\
&\\
&$T^2\myvec{1\\-1\\3}=T\myvec{2\\-2\\-3}=\myvec{4\\-4\\3}$\\
&\\
&Degree of minimal polynomial is 2 therefore k=2\\
&\\

&From Theorem 2\\
&\\
&$\mathbb{Z}\brak{\vec{a};T}$ spans $\{\vec{a},T\vec{a},T^2\vec{a}\}=\myvec{1&2&4\\-1&-2&-4\\3&-3&3}$\\
&which is linearly dependent matrix.$g\brak{T}=$ det$\brak{\mathbb{Z}\brak{\vec{a};T}}=$0\\
&\\
&$\myvec{4\\-4\\3}$ is a linear combination of $\myvec{1\\-1\\3}$ and $\myvec{2\\-2\\3}$.\\
&\\
&\\

&\\
&\\
\hline
&\\
& Therefore from Theorem 3\\
&\\
&Cyclic subspace of $\mathbb{Z}\brak{\vec{a};T}$ spans $\{\vec{a} , T\vec{a}\}$\\
&\\
&Hence $T$-cycle subspace generated by $\myvec{1\\-1\\3}$\\
&\\
&= span $\brak{\myvec{1\\-1\\3},\myvec{2\\-2\\-3}}$ \\

\hline

\caption{Solution Table}
\label{table:2}
\end{longtable}
\end{document}

