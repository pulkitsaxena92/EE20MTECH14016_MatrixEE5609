\documentclass[journal,12pt]{IEEEtran}
\usepackage{longtable}
\usepackage{setspace}
\usepackage{gensymb}
\singlespacing
\usepackage[cmex10]{amsmath}
\newcommand\myemptypage{
	\null
	\thispagestyle{empty}
	\addtocounter{page}{-1}
	\newpage
}
\usepackage{amsthm}
\usepackage{mdframed}
\usepackage{mathrsfs}
\usepackage{txfonts}
\usepackage{stfloats}
\usepackage{bm}
\usepackage{cite}
\usepackage{cases}
\usepackage{subfig}

\usepackage{longtable}
\usepackage{multirow}

\usepackage{enumitem}
\usepackage{mathtools}
\usepackage{steinmetz}
\usepackage{tikz}
\usepackage{circuitikz}
\usepackage{verbatim}
\usepackage{tfrupee}
\usepackage[breaklinks=true]{hyperref}
\usepackage{graphicx}
\usepackage{tkz-euclide}

\usetikzlibrary{calc,math}
\usepackage{listings}
    \usepackage{color}                                            %%
    \usepackage{array}                                            %%
    \usepackage{longtable}                                        %%
    \usepackage{calc}                                             %%
    \usepackage{multirow}                                         %%
    \usepackage{hhline}                                           %%
    \usepackage{ifthen}                                           %%
    \usepackage{lscape}     
\usepackage{multicol}
\usepackage{chngcntr}

\DeclareMathOperator*{\Res}{Res}

\renewcommand\thesection{\arabic{section}}
\renewcommand\thesubsection{\thesection.\arabic{subsection}}
\renewcommand\thesubsubsection{\thesubsection.\arabic{subsubsection}}

\renewcommand\thesectiondis{\arabic{section}}
\renewcommand\thesubsectiondis{\thesectiondis.\arabic{subsection}}
\renewcommand\thesubsubsectiondis{\thesubsectiondis.\arabic{subsubsection}}


\hyphenation{op-tical net-works semi-conduc-tor}
\def\inputGnumericTable{}                                 %%

\lstset{
%language=C,
frame=single, 
breaklines=true,
columns=fullflexible
}
\begin{document}
\onecolumn

\newtheorem{theorem}{Theorem}[section]
\newtheorem{problem}{Problem}
\newtheorem{proposition}{Proposition}[section]
\newtheorem{lemma}{Lemma}[section]
\newtheorem{corollary}[theorem]{Corollary}
\newtheorem{example}{Example}[section]
\newtheorem{definition}[problem]{Definition}

\newcommand{\BEQA}{\begin{eqnarray}}
\newcommand{\EEQA}{\end{eqnarray}}
\newcommand{\define}{\stackrel{\triangle}{=}}
\bibliographystyle{IEEEtran}
\raggedbottom
\setlength{\parindent}{0pt}
\providecommand{\mbf}{\mathbf}
\providecommand{\pr}[1]{\ensuremath{\Pr\left(#1\right)}}
\providecommand{\qfunc}[1]{\ensuremath{Q\left(#1\right)}}
\providecommand{\sbrak}[1]{\ensuremath{{}\left[#1\right]}}
\providecommand{\lsbrak}[1]{\ensuremath{{}\left[#1\right.}}
\providecommand{\rsbrak}[1]{\ensuremath{{}\left.#1\right]}}
\providecommand{\brak}[1]{\ensuremath{\left(#1\right)}}
\providecommand{\lbrak}[1]{\ensuremath{\left(#1\right.}}
\providecommand{\rbrak}[1]{\ensuremath{\left.#1\right)}}
\providecommand{\cbrak}[1]{\ensuremath{\left\{#1\right\}}}
\providecommand{\lcbrak}[1]{\ensuremath{\left\{#1\right.}}
\providecommand{\rcbrak}[1]{\ensuremath{\left.#1\right\}}}
\theoremstyle{remark}
\newtheorem{rem}{Remark}
\newcommand{\sgn}{\mathop{\mathrm{sgn}}}
\providecommand{\abs}[1]{\left\vert#1\right\vert}
\providecommand{\res}[1]{\Res\displaylimits_{#1}} 
\providecommand{\norm}[1]{\left\lVert#1\right\rVert}
%\providecommand{\norm}[1]{\lVert#1\rVert}
\providecommand{\mtx}[1]{\mathbf{#1}}
\providecommand{\mean}[1]{E\left[ #1 \right]}
\providecommand{\fourier}{\overset{\mathcal{F}}{ \rightleftharpoons}}
%\providecommand{\hilbert}{\overset{\mathcal{H}}{ \rightleftharpoons}}
\providecommand{\system}{\overset{\mathcal{H}}{ \longleftrightarrow}}
	%\newcommand{\solution}[2]{\textbf{Solution:}{#1}}
\newcommand{\solution}{\noindent \textbf{Solution: }}
\newcommand{\cosec}{\,\text{cosec}\,}
\providecommand{\dec}[2]{\ensuremath{\overset{#1}{\underset{#2}{\gtrless}}}}
\newcommand{\myvec}[1]{\ensuremath{\begin{pmatrix}#1\end{pmatrix}}}
\newcommand{\mydet}[1]{\ensuremath{\begin{vmatrix}#1\end{vmatrix}}}
\numberwithin{equation}{subsection}
\makeatletter
\@addtoreset{figure}{problem}
\makeatother
\let\StandardTheFigure\thefigure
\let\vec\mathbf
\renewcommand{\thefigure}{\theproblem}
\def\putbox#1#2#3{\makebox[0in][l]{\makebox[#1][l]{}\raisebox{\baselineskip}[0in][0in]{\raisebox{#2}[0in][0in]{#3}}}}
     \def\rightbox#1{\makebox[0in][r]{#1}}
     \def\centbox#1{\makebox[0in]{#1}}
     \def\topbox#1{\raisebox{-\baselineskip}[0in][0in]{#1}}
     \def\midbox#1{\raisebox{-0.5\baselineskip}[0in][0in]{#1}}
\vspace{3cm}
\title{Challenging Problem}
\author{Pulkit Saxena}
\maketitle

\renewcommand{\thefigure}{\theenumi}
\renewcommand{\thetable}{\theenumi}

\section{\textbf{Problem}}
Let $W_1$ and $W_2$ be subspaces of a finite-dimensional vector space $\mathbb V$. Prove that
\begin{enumerate}
    \item $(W_1 + W_2)^0 = W_1^0 \cap W_2^0$
    \item $(W_1 \cap W_2)^0 = W_1^0 + W_2^0$
\end{enumerate}
\section{\textbf{Solutions}}
\renewcommand{\thetable}{1}
\begin{longtable}{|l|l|}
\hline
\multirow{3}{*}{Proof of $W_1+W_2={W_1\cup W_2}$ } & \\
& If $W_1$ and $W_2$ are subspace of vector space $\mathbb{V}$ over a Field F then\\
&1) $W_1+W_2$ is a subspace of $\mathbb{V}$\\
&2)span$\brak{W_1\cup W_2}=W_1+W_2$ or $W_1+W_2={W_1\cup W_2}$\\
&\\
&\textbf{Proof of $W_1+W_2$ is a subspace of $\mathbb{V}$ }\\
&\\
&Let $a_1 ,a_2 \in W_1$ and $b_1,b_2 \in W_2$ and two scalar $\alpha$ and $\beta \in$ F\\
&$\alpha\vec{a}+\beta\vec{b}=\alpha(a_1+a_2)+\beta(b_1+b_2)=\alpha a_1+\beta b_1+\alpha a_2+\beta b_2$\\
&$\implies W_1+W_2\in\mathbb{V}$\\
&\\
&\textbf{Proof of span$\brak{W_1\cup W_2}=W_1+W_2$ or span$(W_1+W_2)={W_1\cup W_2}$ }\\
&\\
&$0\in W_2$\\
&Let $a_1\in W_1$\\
&$a_1=a_1+0\implies a_1\in W_1+W_2$ thus $W_1\subseteq W_1+W_2$ Similarly $W_2\subseteq W_1+W_2$\\
&\\
&\textbf{We need to show $W_1+W_2\subseteq$ span$(W_1\cup W_2)$ and span$(W_1\cup W_2)\subseteq W_1+W_2$}\\ 
&\\
&Let $a=a_1+b_1$ be any element of $W_1+W_2$\\
&Then $a_1\in W_1$ and $a_2\in W_2$\\
&therefore $a_1\in W_1\cup W_2$ and $b_1\in W_1\cup W_2$\\
&We can write $a_1+b_1=1a_1+1b_1$ Thus $a_1$ and $b_1$ is a linear combination \\
&of finite number of elements $a_1,b_1\in W_1\cup W_2$\\
&$\implies a_1+b_1\in$ span$(W_1 \cup W_2)$\\
&$\implies W_1+W_2\in$ span$(W_1 \cup W_2)$\\
&\\
&\textbf{Now to prove span$(W_1\cup W_2)\subseteq W_1+W_2$}\\
&span$(W_1\cup W_2)$ is the smallest subspace containing $W_1\cup W_2$ and $W_1+W_2 $is a\\ 
&subspace of containing $W_1\cup W_2\implies span(W_1\cup W_2)\subseteq W_1 + W_2$\\
&\\
&Hence we proved $W_1+W_2=W_1\cap W_2$.\\
&From this proof the above equation can be modified as De-Morgan's law.\\
&\\
\hline
&\\
\newpage
\hline
&\\
Proof $(W_1 + W_2)^0 = W_1^0 \cap W_2^0$ & From the above proof we can Modify as De-Morgan's Law as \\
&\\
&$(W_1\cup W_2)' = W_1' \cap W_2'$\\
&\\
&Let $x\in(W_1\cup W_2)'$\\
&\\
&$\implies x \not\in (W_1\cup W_2)$\\
&\\
&$\implies x \not\in W_1$ and $x \not\in W_2)$\\
&\\
&$\implies x \in W_1'$ and $x \in W_2')$\\
&\\
&$\implies x\in W_1'\cap W_2'$\\
&\\
& $\implies (W_1\cup W_2)' = W_1' \cap W_2'$\\
&\\
&Therefore $(W_1 + W_2)^0 = W_1^0 \cap W_2^0$. Hence proved.\\
&\\
\hline
&\\
Proof $(W_1 \cap W_2)^0 = W_1^0 + W_2^0$ & From the above proof we can Modify as De-Morgan's Law as \\
&\\
&$(W_1\cap W_2)' = W_1' \cup W_2'$\\
&\\
&Let $x\in(W_1\cap W_2)'$\\
&\\
&$\implies x \not\in (W_1\cap W_2)$\\
&\\
&$\implies x \not\in W_ 1$ OR $x \not\in W_2)$\\
&\\
&$\implies x \in W_ 1'$ OR $x \in W_2')$\\
&\\
&$\implies x\in W_1'\cup W_2'$\\
&\\
&$\implies(W_1 \cup W_2)' = W_1' \cup W_2'$.\\ 
&\\
&Therefore $(W_1 \cap W_2)^0 = W_1^0 + W_2^0$Hence proved.\\
&\\
\hline

\caption{Solution Table}















\label{table:1}
\end{longtable}

\end{document}



