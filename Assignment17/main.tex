\documentclass[journal,12pt]{IEEEtran}
\usepackage{longtable}
\usepackage{setspace}
\usepackage{gensymb}
\singlespacing
\usepackage[cmex10]{amsmath}
\newcommand\myemptypage{
	\null
	\thispagestyle{empty}
	\addtocounter{page}{-1}
	\newpage
}
\usepackage{amsthm}
\usepackage{mdframed}
\usepackage{mathrsfs}
\usepackage{txfonts}
\usepackage{stfloats}
\usepackage{bm}
\usepackage{cite}
\usepackage{cases}
\usepackage{subfig}

\usepackage{longtable}
\usepackage{multirow}

\usepackage{enumitem}
\usepackage{mathtools}
\usepackage{steinmetz}
\usepackage{tikz}
\usepackage{circuitikz}
\usepackage{verbatim}
\usepackage{tfrupee}
\usepackage[breaklinks=true]{hyperref}
\usepackage{graphicx}
\usepackage{tkz-euclide}

\usetikzlibrary{calc,math}
\usepackage{listings}
    \usepackage{color}                                            %%
    \usepackage{array}                                            %%
    \usepackage{longtable}                                        %%
    \usepackage{calc}                                             %%
    \usepackage{multirow}                                         %%
    \usepackage{hhline}                                           %%
    \usepackage{ifthen}                                           %%
    \usepackage{lscape}     
\usepackage{multicol}
\usepackage{chngcntr}

\DeclareMathOperator*{\Res}{Res}

\renewcommand\thesection{\arabic{section}}
\renewcommand\thesubsection{\thesection.\arabic{subsection}}
\renewcommand\thesubsubsection{\thesubsection.\arabic{subsubsection}}

\renewcommand\thesectiondis{\arabic{section}}
\renewcommand\thesubsectiondis{\thesectiondis.\arabic{subsection}}
\renewcommand\thesubsubsectiondis{\thesubsectiondis.\arabic{subsubsection}}


\hyphenation{op-tical net-works semi-conduc-tor}
\def\inputGnumericTable{}                                 %%

\lstset{
%language=C,
frame=single, 
breaklines=true,
columns=fullflexible
}
\begin{document}
\onecolumn

\newtheorem{theorem}{Theorem}[section]
\newtheorem{problem}{Problem}
\newtheorem{proposition}{Proposition}[section]
\newtheorem{lemma}{Lemma}[section]
\newtheorem{corollary}[theorem]{Corollary}
\newtheorem{example}{Example}[section]
\newtheorem{definition}[problem]{Definition}

\newcommand{\BEQA}{\begin{eqnarray}}
\newcommand{\EEQA}{\end{eqnarray}}
\newcommand{\define}{\stackrel{\triangle}{=}}
\bibliographystyle{IEEEtran}
\raggedbottom
\setlength{\parindent}{0pt}
\providecommand{\mbf}{\mathbf}
\providecommand{\pr}[1]{\ensuremath{\Pr\left(#1\right)}}
\providecommand{\qfunc}[1]{\ensuremath{Q\left(#1\right)}}
\providecommand{\sbrak}[1]{\ensuremath{{}\left[#1\right]}}
\providecommand{\lsbrak}[1]{\ensuremath{{}\left[#1\right.}}
\providecommand{\rsbrak}[1]{\ensuremath{{}\left.#1\right]}}
\providecommand{\brak}[1]{\ensuremath{\left(#1\right)}}
\providecommand{\lbrak}[1]{\ensuremath{\left(#1\right.}}
\providecommand{\rbrak}[1]{\ensuremath{\left.#1\right)}}
\providecommand{\cbrak}[1]{\ensuremath{\left\{#1\right\}}}
\providecommand{\lcbrak}[1]{\ensuremath{\left\{#1\right.}}
\providecommand{\rcbrak}[1]{\ensuremath{\left.#1\right\}}}
\theoremstyle{remark}
\newtheorem{rem}{Remark}
\newcommand{\sgn}{\mathop{\mathrm{sgn}}}
\providecommand{\abs}[1]{\left\vert#1\right\vert}
\providecommand{\res}[1]{\Res\displaylimits_{#1}} 
\providecommand{\norm}[1]{\left\lVert#1\right\rVert}
%\providecommand{\norm}[1]{\lVert#1\rVert}
\providecommand{\mtx}[1]{\mathbf{#1}}
\providecommand{\mean}[1]{E\left[ #1 \right]}
\providecommand{\fourier}{\overset{\mathcal{F}}{ \rightleftharpoons}}
%\providecommand{\hilbert}{\overset{\mathcal{H}}{ \rightleftharpoons}}
\providecommand{\system}{\overset{\mathcal{H}}{ \longleftrightarrow}}
	%\newcommand{\solution}[2]{\textbf{Solution:}{#1}}
\newcommand{\solution}{\noindent \textbf{Solution: }}
\newcommand{\cosec}{\,\text{cosec}\,}
\providecommand{\dec}[2]{\ensuremath{\overset{#1}{\underset{#2}{\gtrless}}}}
\newcommand{\myvec}[1]{\ensuremath{\begin{pmatrix}#1\end{pmatrix}}}
\newcommand{\mydet}[1]{\ensuremath{\begin{vmatrix}#1\end{vmatrix}}}
\numberwithin{equation}{subsection}
\makeatletter
\@addtoreset{figure}{problem}
\makeatother
\let\StandardTheFigure\thefigure
\let\vec\mathbf
\renewcommand{\thefigure}{\theproblem}
\def\putbox#1#2#3{\makebox[0in][l]{\makebox[#1][l]{}\raisebox{\baselineskip}[0in][0in]{\raisebox{#2}[0in][0in]{#3}}}}
     \def\rightbox#1{\makebox[0in][r]{#1}}
     \def\centbox#1{\makebox[0in]{#1}}
     \def\topbox#1{\raisebox{-\baselineskip}[0in][0in]{#1}}
     \def\midbox#1{\raisebox{-0.5\baselineskip}[0in][0in]{#1}}
\vspace{3cm}
\title{Assignment 17}
\author{Pulkit Saxena}
\maketitle

\renewcommand{\thefigure}{\theenumi}
\renewcommand{\thetable}{\theenumi}

\section{\textbf{Problem UGCDEC2015 Q76}}
 Let $\vec{A}$ be an m x n real matrix and $\vec{b}\in \mathbb{R}^m$ with $b\neq 0$.
\begin{enumerate}
    \item The set of all real solutions of $\vec{A}x=\vec{b}$ is a vector space.\\
    \item If u nd v are two solutions of $\vec{A}x=\vec{b}$ then $\lambda u  +\brak{1-\lambda}v$ is also a solution of $\vec{A}x=\vec{b}$\\
    \item For any two solutions u and v of $\vec{A}x=\vec{b}$, the linear combination $\lambda u$ + $\brak{1-\lambda}v$ is also a solution of $\vec{A}x=\vec{b}$ only when $0\leq\lambda\leq1.$\\
    \item If rank of $\vec{A}$ is n ,then $\vec{A}x=\vec{b}$ has at most one solution.\ 
    \end{enumerate}
\section{\textbf{Solutions}}
\renewcommand{\thetable}{1}
\begin{longtable}{|l|l|}
\hline
\multirow{3}{*}{Option 1} & \\
&Suppose $\mathbb{V}$ is the vector space defined as $\mathbb{V}=\{\vec{x}:\vec{A}\vec{x}=\vec{b}$ , $\mathbb{R}^n\xrightarrow{}\mathbb{R}^m\}$\\
&\\
& $\vec{v}$ and $\vec{u}$ are the solution to the equation $\vec{A}\vec{x}=\vec{b}$  such that $\vec{u}$ and  $\vec{v}\in \mathbb{V}$\\
&\\
&$\vec{A}\vec{u}=\vec{b}\quad\vec{A}\vec{v}=\vec{b}$\\
&\\
&Checking Closure under vector addition\\
&\\
&$\vec{A}\brak{\vec{u+v}}=\vec{A}\vec{u}+\vec{A}\vec{v}=\vec{b}+\vec{b}=2\vec{b}\neq\vec{b}$\\
&\\
&Which is enclosed under vector addition if and only if $\vec{b}=\vec{0}$.But here given $\vec{b}\neq0$ means $\vec{0} \not\in \mathbb{V}$\\
&\\
&Hence does not satisfy requirements of vector space.\\
&\\
&Hence option 1 is incorrect.\\
&\\
\hline
&\\
Option 2 &$\textbf{Proof 1:}$\\
&\\
&If $\vec{u}$ and $\vec{v}$ are the two solution of $\vec{A}x=\vec{b}$ \\
&\\
&$\vec{A}\vec{u}=\vec{b}\quad\vec{A}\vec{v}=\vec{b}$\\
&\\
&For $\lambda \vec{u}  +\brak{1-\lambda}\vec{v}$ to be a solution of $\vec{A}x=\vec{b}$ ,it must satisfy this equation.\\
&\\
& $\vec{A}\brak{\lambda\vec{u}  +\brak{1-\lambda}\vec{v}}=\vec{b} \implies\vec{A}\lambda\vec{u} + \vec{A}\brak{1-\lambda}\vec{v}=\vec{b}\implies\vec{A}\lambda\vec{u}+\vec{A}\vec{v}-\vec{A}\lambda\vec{v}=\vec{b}$\\
&\\
&$\vec{b}\lambda + \vec{A}\vec{v} -\vec{b}\lambda=\vec{b}\implies\vec{A}\vec{v}=\vec{b}$\\
&\\
&Which satisfies the equation therefore $\lambda \vec{u}  +\brak{1-\lambda}\vec{v}$ is the solution of $\vec{A}x=\vec{b}$ for any $\lambda$\\
&\\
&Since the $\lambda$ term cancels out therefore vaild for $\lambda\in\mathbb{R}$.\\
&\\
&$\textbf{Proof 2 (Through affine Subspace with an Example)}$:-\\
&\\
&Let us suppose the two solution $\vec{u}$ and $\vec{v}$ be the points on the line given by the equation $\vec{A}x=\vec{b}$\\
&\\
&Let the Line joining these two points is given as\\
&\\
&$\vec{l}=\vec{u}-\vec{v}$ is line parallel to the given line $\vec{A}x=\vec{b}$\\
&\\
&Therefore $\vec{v}$ belongs to solution set and is independent to other linearly independent vectors of $\vec{l}$\\
&\\
&$\vec{x}=\vec{v}+\lambda\vec{l}$ for $\lambda\in\mathbb{R}$ on substuting $\vec{l}$\\
&\\
&$\vec{x}=\vec{v}+\lambda\brak{\vec{u}-\vec{v}}=\vec{v}+\lambda\vec{u}-\lambda\vec{v}=\vec{v}\brak{1-\lambda}+\lambda\vec{u}$\\
&\\
&Hence $\vec{v}\brak{1-\lambda}+\lambda\vec{u}$ is also the solution of the equation $\vec{A}\vec{x}=\vec{b}$ for $\lambda\in\mathbb{R}$.\\
&\\
&Hence Option 2 is correct.\\
&\\
\hline
&\\
Option 3 &Since in Option 2 we have proved that $\vec{v}\brak{1-\lambda}+\lambda\vec{u}$  is a solution for $\vec{A}\vec{x}=\vec{b}$  for any $\lambda\in\mathbb{R}$\\
&\\
&therefore $\lambda$ can be any real value but in option 3 there is restriction on $\lambda$ which is incorrect.\\
&\\
&Hence option 3 is incorrect\\
\hline
&\\
Option 4
&$\vec{A}_{mxn}\vec{x}_{nx1}=\vec{b}_{mx1}$\\
&\\
&If $\vec{A}$ has Full column rank$\brak{\vec{A}}=n$ then there exist one pivot in each columns \\
&\\
&and there exists no free variables thus $\vec{N\brak{A}}=\vec{0}$ so the only solution to $\vec{A}\vec{x}=\vec{0}$ is $\vec{x}=\vec{0}$.\\ 
&\\
&So the solution to $\vec{A}\vec{x}=\vec{b}$\\
&\\
&$\vec{x}=\vec{x_p}$ unique solution exists if it exist.It can be either 0 or 1.\\
&\\
&Hence at most 1 solution is possible .\\
&\\
&\textbf{Proof with example}\\
&\\
&Let $\vec{A}=\myvec{1&3\\2&1\\6&1\\5&1}_{4x2}\xleftrightarrow{RREF}\myvec{1&0\\0&1\\0&0\\0&0}$ Hence $n=2$ pivot columns at both column position \\ 
&\\
&$\myvec{1&0\\0&1\\0&0\\0&0}\myvec{x_1\\x_2}=\myvec{b_1\\b_2\\b_3\\b_4}$ Hence no solution possible  as no combination of $\vec{x}$ can gives the solution except\\
&\\
&$\vec{x}=\myvec{0\\0}$ only if $\vec{b}=\vec{0} \implies \myvec{1&3\\2&1\\6&1\\5&1}\myvec{0\\0}=\myvec{0\\0\\0\\0}$ \textbf{OR}\\
&\\
&$\vec{x}=\myvec{1\\1}$ only if $\vec{b}$ is addition of columns of $\vec{A}$ $\implies \myvec{1&3\\2&1\\6&1\\5&1}\myvec{1\\1}=\myvec{4\\3\\7\\6}$\\
&\\
&Hence either no solution possible or one solution possile.\\&Therefore we say at most one solution possible.\\
&\\
&Option 4 is correct.\\
&\\
\hline
&\\
Answers & Option 2 and Option 4 are correct\\
&\\
\hline
\caption{Solution}
\label{table:1}
\end{longtable}

\end{document}


