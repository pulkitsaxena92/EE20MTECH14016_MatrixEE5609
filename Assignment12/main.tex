\documentclass[journal,12pt]{IEEEtran}
\usepackage{longtable}
\usepackage{setspace}
\usepackage{gensymb}
\singlespacing
\usepackage[cmex10]{amsmath}
\newcommand\myemptypage{
	\null
	\thispagestyle{empty}
	\addtocounter{page}{-1}
	\newpage
}
\usepackage{amsthm}
\usepackage{mdframed}
\usepackage{mathrsfs}
\usepackage{txfonts}
\usepackage{stfloats}
\usepackage{bm}
\usepackage{cite}
\usepackage{cases}
\usepackage{subfig}

\usepackage{longtable}
\usepackage{multirow}

\usepackage{enumitem}
\usepackage{mathtools}
\usepackage{steinmetz}
\usepackage{tikz}
\usepackage{circuitikz}
\usepackage{verbatim}
\usepackage{tfrupee}
\usepackage[breaklinks=true]{hyperref}
\usepackage{graphicx}
\usepackage{tkz-euclide}

\usetikzlibrary{calc,math}
\usepackage{listings}
    \usepackage{color}                                            %%
    \usepackage{array}                                            %%
    \usepackage{longtable}                                        %%
    \usepackage{calc}                                             %%
    \usepackage{multirow}                                         %%
    \usepackage{hhline}                                           %%
    \usepackage{ifthen}                                           %%
    \usepackage{lscape}     
\usepackage{multicol}
\usepackage{chngcntr}

\DeclareMathOperator*{\Res}{Res}

\renewcommand\thesection{\arabic{section}}
\renewcommand\thesubsection{\thesection.\arabic{subsection}}
\renewcommand\thesubsubsection{\thesubsection.\arabic{subsubsection}}

\renewcommand\thesectiondis{\arabic{section}}
\renewcommand\thesubsectiondis{\thesectiondis.\arabic{subsection}}
\renewcommand\thesubsubsectiondis{\thesubsectiondis.\arabic{subsubsection}}


\hyphenation{op-tical net-works semi-conduc-tor}
\def\inputGnumericTable{}                                 %%

\lstset{
%language=C,
frame=single, 
breaklines=true,
columns=fullflexible
}
\begin{document}
\onecolumn

\newtheorem{theorem}{Theorem}[section]
\newtheorem{problem}{Problem}
\newtheorem{proposition}{Proposition}[section]
\newtheorem{lemma}{Lemma}[section]
\newtheorem{corollary}[theorem]{Corollary}
\newtheorem{example}{Example}[section]
\newtheorem{definition}[problem]{Definition}

\newcommand{\BEQA}{\begin{eqnarray}}
\newcommand{\EEQA}{\end{eqnarray}}
\newcommand{\define}{\stackrel{\triangle}{=}}
\bibliographystyle{IEEEtran}
\raggedbottom
\setlength{\parindent}{0pt}
\providecommand{\mbf}{\mathbf}
\providecommand{\pr}[1]{\ensuremath{\Pr\left(#1\right)}}
\providecommand{\qfunc}[1]{\ensuremath{Q\left(#1\right)}}
\providecommand{\sbrak}[1]{\ensuremath{{}\left[#1\right]}}
\providecommand{\lsbrak}[1]{\ensuremath{{}\left[#1\right.}}
\providecommand{\rsbrak}[1]{\ensuremath{{}\left.#1\right]}}
\providecommand{\brak}[1]{\ensuremath{\left(#1\right)}}
\providecommand{\lbrak}[1]{\ensuremath{\left(#1\right.}}
\providecommand{\rbrak}[1]{\ensuremath{\left.#1\right)}}
\providecommand{\cbrak}[1]{\ensuremath{\left\{#1\right\}}}
\providecommand{\lcbrak}[1]{\ensuremath{\left\{#1\right.}}
\providecommand{\rcbrak}[1]{\ensuremath{\left.#1\right\}}}
\theoremstyle{remark}
\newtheorem{rem}{Remark}
\newcommand{\sgn}{\mathop{\mathrm{sgn}}}
\providecommand{\abs}[1]{\left\vert#1\right\vert}
\providecommand{\res}[1]{\Res\displaylimits_{#1}} 
\providecommand{\norm}[1]{\left\lVert#1\right\rVert}
%\providecommand{\norm}[1]{\lVert#1\rVert}
\providecommand{\mtx}[1]{\mathbf{#1}}
\providecommand{\mean}[1]{E\left[ #1 \right]}
\providecommand{\fourier}{\overset{\mathcal{F}}{ \rightleftharpoons}}
%\providecommand{\hilbert}{\overset{\mathcal{H}}{ \rightleftharpoons}}
\providecommand{\system}{\overset{\mathcal{H}}{ \longleftrightarrow}}
	%\newcommand{\solution}[2]{\textbf{Solution:}{#1}}
\newcommand{\solution}{\noindent \textbf{Solution: }}
\newcommand{\cosec}{\,\text{cosec}\,}
\providecommand{\dec}[2]{\ensuremath{\overset{#1}{\underset{#2}{\gtrless}}}}
\newcommand{\myvec}[1]{\ensuremath{\begin{pmatrix}#1\end{pmatrix}}}
\newcommand{\mydet}[1]{\ensuremath{\begin{vmatrix}#1\end{vmatrix}}}
\numberwithin{equation}{subsection}
\makeatletter
\@addtoreset{figure}{problem}
\makeatother
\let\StandardTheFigure\thefigure
\let\vec\mathbf
\renewcommand{\thefigure}{\theproblem}
\def\putbox#1#2#3{\makebox[0in][l]{\makebox[#1][l]{}\raisebox{\baselineskip}[0in][0in]{\raisebox{#2}[0in][0in]{#3}}}}
     \def\rightbox#1{\makebox[0in][r]{#1}}
     \def\centbox#1{\makebox[0in]{#1}}
     \def\topbox#1{\raisebox{-\baselineskip}[0in][0in]{#1}}
     \def\midbox#1{\raisebox{-0.5\baselineskip}[0in][0in]{#1}}
\vspace{3cm}
\title{Assignment 12}
\author{Pulkit Saxena}
\maketitle

\renewcommand{\thefigure}{\theenumi}
\renewcommand{\thetable}{\theenumi}

\section{\textbf{Problem ugcjune2017 Q75}}
Which of the following 3x3 matrices are diagonizable over $\mathbb{R}?$\\
\\$1.\myvec{1&2&3\\0&4&5\\0&0&6}\quad 2. \myvec{0&1&0\\-1&0&0\\0&0&1}\quad 3.\myvec{1&2&3\\2&1&4\\3&4&1}\quad 4.\myvec{0&1&2\\0&0&1\\0&0&0}$\\

\section{\textbf{Explanation}}
\renewcommand{\thetable}{1}
\begin{longtable}{|l|l|}
\hline
\multirow{3}{*}{Test for diagonalizability} & \\
& Let $\vec{W}_{i}$ be the eigenspace corresponding to eigenvalue $\lambda_{i}$  of $\vec{A}$\\
&\\
& $1)\vec{A}$ is diagonalizable \\
&\\
& $2)$ characteristic polynomial of $\vec{A}$ is \\
& f = $(\vec{x}-\lambda_1)^{d_1}....(\vec{x}-\lambda_k)^{d_k}$ and $dim(\vec{W}_i) = d_i $\\
&\\
& $3) \sum_{i=1}^{k}\vec{W_i}=n$\\
&\\
\hline
\multirow{3}{*}{Concept} & \\
&
A linear operator $\vec{A}$ on a $n$-dimensional space $\mathbb{V}$ is\\ 
&\\ for diagonalization
& diagonalizable , if and only if $\vec{A}$ has $n$ distinct \\
&\\
& characteristic vectors or null spaces corresponding to the characteristic values\\
\hline
\caption{Illustration of theorem.}
\label{table:1}
\end{longtable}
\newpage
\section{\textbf{Solution}}
\renewcommand{\thetable}{2}
\begin{longtable}{|l|l|}
\hline
\multirow{3}{*}{Option A} & \\
& Given matrix is  \\
&\\
& $\vec{A}$=$\myvec{1&2&3\\0&4&5\\0&0&6}$\\
&\\
\hline
\multirow{3}{*}{Finding Characteristics} & \\
&
Characteristics polynomial of the matrix $\vec{A}$ is $det(xI-A)$\\ 
polynomial
& $\det(xI-A)$= $\left|
                \begin{array}{ccc}
                (x-1) & -3 & -2\\
                0 & (x-4) & -5\\
                0 & 0 & x-6
                \end{array} \right|$  \\
&\\
& Characteristic Polynomial = $(x-1)(x-4)(x-6)$\\
&\\
\hline
\multirow{3}{*}{Testing diagonalizability over $\mathbb{R}$} & \\
& 1) As the characteristics  polynomial is product of linear factors\\
&over $\mathbb{R}$ .\\
&\\
&2) To find characteristic values of the operator $\det(xI-A) = 0$ which gives  \\
& $\lambda_1= 1 , \lambda_2= 4, \lambda_3= 6$\\
&\\
& Thus over $\mathbb{R}$ matrix $\vec{A}$ has three distinct characteristic values.\\
&There will be atleast one characteristics vector i.e., one\\ & dimension with each characteristics value .\\
&Thus $dim W_i$ = $d_i$\\
&\\
&3) $\sum_{i} \vec{W_i} = n = 3$ , which is equal to $dim$ of $A$.\\ 
&\\
\hline
\multirow{3}{*}{Conclusion on Option A} & \\
& Option A satisfy all three condition of Diagonalizability over $\mathbb{R}$. \\
&\\

\hline \hline
\multirow{3}{*}{Option B} & \\
& Given matrix is  \\
&\\
& $\vec{A}$=$\myvec{0&1&0\\-1&0&0\\0&0&1}$\\
&\\
\hline
\multirow{3}{*}{Finding Characteristics} & \\
&
Characteristics polynomial of the matrix $\vec{A}$ is $det(xI-A)$\\ 
polynomial
& $\det(xI-A)$= $\left|
                \begin{array}{ccc}
                x & -1 & 0\\
                1 & x & 0\\
                0 & 0 & x-1
                \end{array} \right|$  \\
&\\
& Characteristic Polynomial = $(x-1)(x+i)(x-i)$\\
&\\
\hline
\multirow{3}{*}{Testing diagonalizability over $\mathbb{R}$} & \\
& 1) As the characteristics  polynomial is not the product of linear factors\\
&over $\mathbb{R}$ . Thus $\vec{A}$ is not diagonalizable over $\mathbb{R}$.\\

&\\
\hline
\multirow{3}{*}{Conclusion on Option B} & \\
& Option B does not satisfy condition 1. \\
&\\
\hline \hline
\multirow{3}{*}{Option C} & \\
& Given matrix is  \\
&\\
& $\vec{A}$=$\myvec{1&2&3\\2&1&4\\3&4&1}$\\
&\\
\hline
\multirow{3}{*}{Finding Characteristics} & \\
&
Characteristics polynomial of the matrix $\vec{A}$ is $det(xI-A)$\\ 
polynomial
& $\det(xI-A)$= $\left|
                \begin{array}{ccc}
                (x-1) & -2 & -3\\
                -2 & (x-1) & -4\\
                -3 & -4 & x-1
                \end{array} \right|$  \\
&\\
& Characteristic Polynomial = $(x+3.19)(x+0.877)(x-7.07)$\\
&\\
\hline
\multirow{3}{*}{Testing diagonalizability over $\mathbb{R}$} & \\
& 1) As the characteristics  polynomial is product of linear factors\\
&over $\mathbb{R}$ .\\
&\\
&2) To find characteristic values of the operator $\det(xI-A) = 0$ which gives  \\
& $\lambda_1= -3.19 , \lambda_2= -0.887, \lambda_3= 7.07$\\
&\\
& Thus over $\mathbb{R}$ matrix $\vec{A}$ has three distinct characteristic values.\\
&There will be atleast one characteristics vector i.e., one\\ & dimension with each characteristics value .\\
&Thus $dim W_i$ = $d_i$\\
&\\
&3) $\sum_{i} \vec{W_i} = n = 3$ , which is equal to $dim$ of $A$.\\ 
&\\
\hline
\multirow{3}{*}{Conclusion on Option C} & \\
& Option C satisfy all three condition of Diagonalizability over $\mathbb{R}$. \\
&\\
\hline\hline
\multirow{3}{*}{Option D} & \\
& Given matrix is  \\
&\\
& $\vec{A}$=$\myvec{0&1&2\\0&0&1\\0&0&0}$\\
&\\
\hline
\multirow{3}{*}{Finding Characteristics} & \\
&
Characteristics polynomial of the matrix $\vec{A}$ is $det(xI-A)$\\ 
polynomial
& $\det(xI-A)$= $\left|
                \begin{array}{ccc}
                x & -1 & -2\\
                0 & x & -1\\
                0 & 0 & x
                \end{array} \right|$  \\
&\\
& Characteristic Polynomial = $(x)(x)(x)=x^3$\\
&\\
\hline
\multirow{3}{*}{Testing diagonalizability over $\mathbb{R}$} & \\
& 1) As the characteristics  polynomial is product of linear factors\\
&over $\mathbb{R}$ .\\
&\\
&2) To find characteristic values of the operator $\det(xI-A) = 0$ \\
& $\lambda_1= 0$\\
&$d_1=3$\\
&$\vec{W}_1=\vec{A}-\lambda_1\vec{I}\implies\myvec{0&1&2\\0&0&1\\0&0&0}-0\myvec{1&0&0\\0&1&0\\0&0&1}=\myvec{0&1&2\\0&0&1\\0&0&0}$\\

&$dim W_1 = 2$\\
&$dim W_i \neq d_i$\\
&Algebric Multiplicity is not equal to Geometric Multiplicity.\\
&\\
\hline
\multirow{3}{*}{Conclusion on Option D} & \\
& Option D  does not satisfy second condition of Diagonalizability. \\
&\\
\hline \hline
\multirow{3}{*}{Answer} & \\
&Option A and Option C are Diagonalizable over $\mathbb{R}$.\\
&\\
\hline



\label{table:2}
\end{longtable}
\end{document}
