\documentclass[journal,12pt,twocolumncolumn]{IEEEtran}

\usepackage{setspace}
\usepackage{gensymb}

\singlespacing


\usepackage[cmex10]{amsmath}

\usepackage{amsthm}

\usepackage{mathrsfs}
\usepackage{txfonts}
\usepackage{stfloats}
\usepackage{bm}
\usepackage{cite}
\usepackage{cases}
\usepackage{subfig}

\usepackage{longtable}
\usepackage{multirow}

\usepackage{enumitem}
\usepackage{mathtools}
\usepackage{steinmetz}
\usepackage{tikz}
\usepackage{circuitikz}
\usepackage{verbatim}
\usepackage{tfrupee}
\usepackage[breaklinks=true]{hyperref}
\usepackage{graphicx}
\usepackage{tkz-euclide}

\usetikzlibrary{calc,math}
\usepackage{listings}
    \usepackage{color}                                            %%
    \usepackage{array}                                            %%
    \usepackage{longtable}                                        %%
    \usepackage{calc}                                             %%
    \usepackage{multirow}                                         %%
    \usepackage{hhline}                                           %%
    \usepackage{ifthen}                                           %%
    \usepackage{lscape}     
\usepackage{multicol}
\usepackage{chngcntr}

\DeclareMathOperator*{\Res}{Res}

\renewcommand\thesection{\arabic{section}}
\renewcommand\thesubsection{\thesection.\arabic{subsection}}
\renewcommand\thesubsubsection{\thesubsection.\arabic{subsubsection}}

\renewcommand\thesectiondis{\arabic{section}}
\renewcommand\thesubsectiondis{\thesectiondis.\arabic{subsection}}
\renewcommand\thesubsubsectiondis{\thesubsectiondis.\arabic{subsubsection}}


\hyphenation{op-tical net-works semi-conduc-tor}
\def\inputGnumericTable{}                                 %%

\lstset{
%language=C,
frame=single, 
breaklines=true,
columns=fullflexible
}
\begin{document}


\newtheorem{theorem}{Theorem}[section]
\newtheorem{problem}{Problem}
\newtheorem{proposition}{Proposition}[section]
\newtheorem{lemma}{Lemma}[section]
\newtheorem{corollary}[theorem]{Corollary}
\newtheorem{example}{Example}[section]
\newtheorem{definition}[problem]{Definition}

\newcommand{\BEQA}{\begin{eqnarray}}
\newcommand{\EEQA}{\end{eqnarray}}
\newcommand{\define}{\stackrel{\triangle}{=}}
\bibliographystyle{IEEEtran}
\providecommand{\mbf}{\mathbf}
\providecommand{\pr}[1]{\ensuremath{\Pr\left(#1\right)}}
\providecommand{\qfunc}[1]{\ensuremath{Q\left(#1\right)}}
\providecommand{\sbrak}[1]{\ensuremath{{}\left[#1\right]}}
\providecommand{\lsbrak}[1]{\ensuremath{{}\left[#1\right.}}
\providecommand{\rsbrak}[1]{\ensuremath{{}\left.#1\right]}}
\providecommand{\brak}[1]{\ensuremath{\left(#1\right)}}
\providecommand{\lbrak}[1]{\ensuremath{\left(#1\right.}}
\providecommand{\rbrak}[1]{\ensuremath{\left.#1\right)}}
\providecommand{\cbrak}[1]{\ensuremath{\left\{#1\right\}}}
\providecommand{\lcbrak}[1]{\ensuremath{\left\{#1\right.}}
\providecommand{\rcbrak}[1]{\ensuremath{\left.#1\right\}}}
\theoremstyle{remark}
\newtheorem{rem}{Remark}
\newcommand{\sgn}{\mathop{\mathrm{sgn}}}
\providecommand{\abs}[1]{\left\vert#1\right\vert}
\providecommand{\res}[1]{\Res\displaylimits_{#1}} 
\providecommand{\norm}[1]{\left\lVert#1\right\rVert}
%\providecommand{\norm}[1]{\lVert#1\rVert}
\providecommand{\mtx}[1]{\mathbf{#1}}
\providecommand{\mean}[1]{E\left[ #1 \right]}
\providecommand{\fourier}{\overset{\mathcal{F}}{ \rightleftharpoons}}
%\providecommand{\hilbert}{\overset{\mathcal{H}}{ \rightleftharpoons}}
\providecommand{\system}{\overset{\mathcal{H}}{ \longleftrightarrow}}
	%\newcommand{\solution}[2]{\textbf{Solution:}{#1}}
\newcommand{\solution}{\noindent \textbf{Solution: }}
\newcommand{\cosec}{\,\text{cosec}\,}
\providecommand{\dec}[2]{\ensuremath{\overset{#1}{\underset{#2}{\gtrless}}}}
\newcommand{\myvec}[1]{\ensuremath{\begin{pmatrix}#1\end{pmatrix}}}
\newcommand{\mydet}[1]{\ensuremath{\begin{vmatrix}#1\end{vmatrix}}}
\numberwithin{equation}{subsection}
\makeatletter
\@addtoreset{figure}{problem}
\makeatother
\let\StandardTheFigure\thefigure
\let\vec\mathbf
\renewcommand{\thefigure}{\theproblem}
\def\putbox#1#2#3{\makebox[0in][l]{\makebox[#1][l]{}\raisebox{\baselineskip}[0in][0in]{\raisebox{#2}[0in][0in]{#3}}}}
     \def\rightbox#1{\makebox[0in][r]{#1}}
     \def\centbox#1{\makebox[0in]{#1}}
     \def\topbox#1{\raisebox{-\baselineskip}[0in][0in]{#1}}
     \def\midbox#1{\raisebox{-0.5\baselineskip}[0in][0in]{#1}}
\vspace{3cm}
\title{Assignment 2}
\author{Pulkit Saxena}
\maketitle
\bigskip
\renewcommand{\thefigure}{1}
\renewcommand{\thetable}{\theenumi}
Download all python codes from 
\begin{lstlisting}
https://github.com/pulkitsaxena92/EE20MTECH14016_MatrixEE5609/tree/master/Assignment2
\end{lstlisting}
%
and python codes from 
%
\begin{lstlisting}
https://github.com/pulkitsaxena92/EE20MTECH14016_MatrixEE5609/tree/master/Assignment2/code
\end{lstlisting}
\section{\textbf{Question}}
If $\vec{A}=\myvec{0 & -\tan\frac{\alpha}{2} \\ \tan\frac{\alpha}{2} & 0 }$ and $\vec{I}$ is identity matrix of order 2 , show that
\begin{align}
  \vec{I}+\vec{A}=\myvec{I-A}\myvec{\cos\alpha & -\sin\alpha \\ \sin\alpha & cos\alpha}
\end{align}

\section{\textbf{Solution}}
Since
\begin{align}
    \vec{I}=\myvec{1 & 0\\ 0 & 1}\\
    \vec{A}=\tan\frac{\alpha}{2}\myvec{\cos{90} & -\sin{90} \\ \sin{90} & \cos{90}}\\
\Vec{I}-\Vec{A}=\myvec{1 & 0\\ 0 & 1}-\tan\frac{\alpha}{2}\myvec{\cos{90} & -\sin{90} \\ \sin{90} & \cos{90}}\\
=\frac{1}{\cos\frac{\alpha}{2}}\myvec{\cos\frac{\alpha}{2} & 0 \\ 0 & \cos\frac{\alpha}{2}}-\frac{\sin\frac{\alpha}{2}}{\cos\frac{\alpha}{2}}\myvec{\cos{90} & -\sin{90} \\ \sin{90} & \cos{90}} \\
=\frac{1}{\cos\frac{\alpha}{2}}\myvec{\cos\frac{\alpha}{2} & 0 \\ 0 & \cos\frac{\alpha}{2}}-\frac{1}{\cos\frac{\alpha}{2}}\myvec{0 & -\sin\frac{\alpha}{2} \\ \sin\frac{\alpha}{2} & 0} \\
=\frac{1}{\cos\frac{\alpha}{2}}\myvec{\cos\frac{\alpha}{2} & \sin\frac{\alpha}{2} \\ -\sin\frac{\alpha}{2} & \cos\frac{\alpha}{2}}
\end{align}
The matrix $\vec{I}-\vec{A}$ is a rotational Matrix with rotation $-\frac{\alpha}{2}$ \\\\
The Matrix $\myvec{\cos\alpha & -\sin\alpha \\ \sin\alpha & \cos\alpha}$ is also a rotational Matrix with an angle $+\alpha$.\\\\
Multiplying two rotational matrices gives the resultant rotational matrix $+\alpha-\frac{\alpha}{2}=+\frac{\alpha}{2}$\\
\begin{align}
  RHS=\myvec{I-A}\myvec{\cos\alpha & -\sin\alpha \\ \sin\alpha & cos\alpha}\\
 =\frac{1}{\cos\frac{\alpha}{2}}\myvec{\cos\frac{\alpha}{2} & \sin\frac{\alpha}{2} \\ -\sin\frac{\alpha}{2} & \cos\frac{\alpha}{2}}\myvec{\cos\alpha & -\sin\alpha \\ \sin\alpha & cos\alpha}\\
   =\frac{1}{\cos\frac{\alpha}{2}}\myvec{\cos\frac{\alpha}{2} & -\sin\frac{\alpha}{2} \\ \sin\frac{\alpha}{2} & \cos\frac{\alpha}{2}}\\
\end{align}
Solving LHS$=\vec{I}+\vec{A}$
\begin{align}
    \vec{I}+\vec{A}=\myvec{1 & \tan\frac{\alpha}{2}\\\tan\frac{\alpha}{2} & 1}\\
    =\frac{1}{\cos\frac{\alpha}{2}}\myvec{\cos\frac{\alpha}{2} & -\sin\frac{\alpha}{2} \\ \sin\frac{\alpha}{2} & \cos\frac{\alpha}{2}}
\end{align}
This term is a rotational Matrix with angle $+\frac{\alpha}{2}$.Hence both sides evaluates to be a rotational matrix with angle $+\frac{\alpha}{2}$.


\end{document}