\documentclass[journal,12pt,twocolumn]{IEEEtran}

\usepackage{setspace}
\usepackage{gensymb}


\singlespacing

\usepackage[cmex10]{amsmath}
%\usepackage{amsthm}
%\interdisplaylinepenalty=2500
%\savesymbol{iint}
%\usepackage{txfonts}
%\restoresymbol{TXF}{iint}
%\usepackage{wasysym}
\usepackage{amsthm}

\usepackage{mathrsfs}
\usepackage{txfonts}
\usepackage{stfloats}
\usepackage{bm}
\usepackage{cite}
\usepackage{cases}
\usepackage{subfig}

\usepackage{longtable}
\usepackage{multirow}

\usepackage{enumitem}
\usepackage{mathtools}
\usepackage{steinmetz}
\usepackage{tikz}
\usepackage{circuitikz}
\usepackage{verbatim}
\usepackage{tfrupee}
\usepackage[breaklinks=true]{hyperref}

\usepackage{tkz-euclide} %loads TikZ and tkz-base

\usetikzlibrary{calc,math}
\usepackage{listings}
    \usepackage{color}                                          
    \usepackage{array}                                          
    \usepackage{longtable}                                      
    \usepackage{calc}                                          
    \usepackage{multirow}                                      
    \usepackage{hhline}                                        
    \usepackage{ifthen}
    \usepackage{lscape}    
\usepackage{multicol}
\usepackage{chngcntr}

\DeclareMathOperator*{\Res}{Res}

\renewcommand\thesection{\arabic{section}}
\renewcommand\thesubsection{\thesection.\arabic{subsection}}
\renewcommand\thesubsubsection{\thesubsection.\arabic{subsubsection}}

\renewcommand\thesectiondis{\arabic{section}}
\renewcommand\thesubsectiondis{\thesectiondis.\arabic{subsection}}
\renewcommand\thesubsubsectiondis{\thesubsectiondis.\arabic{subsubsection}}

\hyphenation{op-tical net-works semi-conduc-tor}
\def\inputGnumericTable{}                                 %%

\lstset{
%language=C,
frame=single,
breaklines=true,
columns=fullflexible
}

\begin{document}

\newtheorem{theorem}{Theorem}[section]
\newtheorem{problem}{Problem}
\newtheorem{proposition}{Proposition}[section]
\newtheorem{lemma}{Lemma}[section]
\newtheorem{corollary}[theorem]{Corollary}
\newtheorem{example}{Example}[section]
\newtheorem{definition}[problem]{Definition}

\newcommand{\BEQA}{\begin{eqnarray}}
\newcommand{\EEQA}{\end{eqnarray}}
\newcommand{\define}{\stackrel{\triangle}{=}}
\bibliographystyle{IEEEtran}
\providecommand{\mbf}{\mathbf}
\providecommand{\pr}[1]{\ensuremath{\Pr\left(#1\right)}}
\providecommand{\qfunc}[1]{\ensuremath{Q\left(#1\right)}}
\providecommand{\sbrak}[1]{\ensuremath{{}\left[#1\right]}}
\providecommand{\lsbrak}[1]{\ensuremath{{}\left[#1\right.}}
\providecommand{\rsbrak}[1]{\ensuremath{{}\left.#1\right]}}
\providecommand{\brak}[1]{\ensuremath{\left(#1\right)}}
\providecommand{\lbrak}[1]{\ensuremath{\left(#1\right.}}
\providecommand{\rbrak}[1]{\ensuremath{\left.#1\right)}}
\providecommand{\cbrak}[1]{\ensuremath{\left\{#1\right\}}}
\providecommand{\lcbrak}[1]{\ensuremath{\left\{#1\right.}}
\providecommand{\rcbrak}[1]{\ensuremath{\left.#1\right\}}}
\theoremstyle{remark}
\newtheorem{rem}{Remark}
\newcommand{\sgn}{\mathop{\mathrm{sgn}}}
\providecommand{\abs}[1]{\left\vert#1\right\vert}
\providecommand{\res}[1]{\Res\displaylimits_{#1}}
\providecommand{\norm}[1]{\left\lVert#1\right\rVert}
%\providecommand{\norm}[1]{\lVert#1\rVert}
\providecommand{\mtx}[1]{\mathbf{#1}}
\providecommand{\mean}[1]{E\left[ #1 \right]}
\providecommand{\fourier}{\overset{\mathcal{F}}{ \rightleftharpoons}}
%\providecommand{\hilbert}{\overset{\mathcal{H}}{ \rightleftharpoons}}
\providecommand{\system}{\overset{\mathcal{H}}{ \longleftrightarrow}}
%\newcommand{\solution}[2]{\textbf{Solution:}{#1}}
\newcommand{\solution}{\noindent \textbf{Solution: }}
\newcommand{\cosec}{\,\text{cosec}\,}
\providecommand{\dec}[2]{\ensuremath{\overset{#1}{\underset{#2}{\gtrless}}}}
\newcommand{\myvec}[1]{\ensuremath{\begin{pmatrix}#1\end{pmatrix}}}
\newcommand{\mydet}[1]{\ensuremath{\begin{vmatrix}#1\end{vmatrix}}}
\numberwithin{equation}{subsection}
\makeatletter
\@addtoreset{figure}{problem}
\makeatother
\let\StandardTheFigure\thefigure
\let\vec\mathbf
\renewcommand{\thefigure}{\theproblem}
\def\putbox#1#2#3{\makebox[0in][l]{\makebox[#1][l]{}\raisebox{\baselineskip}[0in][0in]{\raisebox{#2}[0in][0in]{#3}}}}
     \def\rightbox#1{\makebox[0in][r]{#1}}
     \def\centbox#1{\makebox[0in]{#1}}
     \def\topbox#1{\raisebox{-\baselineskip}[0in][0in]{#1}}
     \def\midbox#1{\raisebox{-0.5\baselineskip}[0in][0in]{#1}}

\title{Assignment5}
\author{Pulkit Saxena}
\maketitle
\newpage

\bigskip
\renewcommand{\thefigure}{\theenumi}
\renewcommand{\thetable}{\theenumi}
\section{\textbf{Problem}}
Find QR decomposition of matrix
\begin{equation}
	\vec{V} = \myvec{12 & -5\\ -5 & 2}
\end{equation}
\section{\textbf{Solution}}
Let $\vec{x}$ and $\vec{y}$ be the column vectors of the given matrix.
\begin{align}
    \vec{x} &= \myvec{12 \\-5 }\\
    \vec{y} &= \myvec{-5 \\ 2}
\end{align}
The column vectors can be expressed as follows,
\begin{align}
    \vec{x} &= k_1\vec{u}_1\label{eq_QR1}\\
    \vec{y} &= r_1\vec{u}_1+k_2\vec{u}_2\label{eq_QR2}
\end{align}
\begin{align}
    k_1 &= \norm{\vec{x}}\label{eq1}\\
    \vec{u}_1 &= \frac{\vec{x}}{k_1}\\
    r_1 &= \frac{\vec{u}_1^T\vec{y}}{\norm{\vec{u}_1}^2}\\
    \vec{u}_2 &= \frac{\vec{y} - r_1 \vec{u}_1}{\norm{\vec{y} - r_1 \vec{u}_1}}\\
    k_2 &= {\vec{u}_2^T\vec{y}}\label{eq2}
\end{align}
The \eqref{eq_QR1} and \eqref{eq_QR2} can be written as, 
\begin{align}
\myvec{\vec{x} & \vec{y}} &= \myvec{\vec{u}_1 & \vec{u}_2}\myvec{k_1 & r_1 \\ 0 & k_2}\label{QRMain}\\
\myvec{\vec{x} & \vec{y}} &= \vec{Q}\vec{R}
\end{align}
Now, $\vec{R}$ is an upper triangular matrix and also,
\begin{align}
\vec{Q}^T\vec{Q}=\vec{I}
\end{align}
Now using equations \eqref{eq1} to \eqref{eq2} we get, 
\begin{align}
    k_1 &= \sqrt{12^2+5^2} = 13\label{eqval1}\\ 
    \vec{u}_1 &= \myvec{\frac{12}{13}\\ \\ \frac{-5}{13}} \\
    r_1 &= \myvec{\frac{12}{13}&&\frac{-5}{13}}\myvec{-5 \\ 2} = -\frac{70}{13}\\ 
    \vec{u}_2 &= \myvec{-\frac{5}{13} \\\\ -\frac{12}{13}} \\
    k_2 &= \myvec{-\frac{5}{13}&&-\frac{12}{13}}\myvec{-5\\2} = \frac{1}{13}\label{eqval2} 
\end{align}
Thus putting the values from \eqref{eqval1} to \eqref{eqval2} in \eqref{QRMain} we obtain QR decomposition,
\begin{align}
    \myvec{12 & -5\\ -5 & 2} =\myvec{\frac{12}{13}&&-\frac{5}{13}\\\\-\frac{5}{13}&&-\frac{12}{13}}\myvec{13 && -\frac{70}{13}\\\\0&&\frac{1}{13}}
\end{align}
which can also be written as,
\begin{align}
  \myvec{12 & -5\\ -5 & 2} =\myvec{-\frac{12}{13}&&\frac{5}{13}\\\\\frac{5}{13}&&\frac{12}{13}}\myvec{-13 && \frac{70}{13}\\\\0&&-\frac{1}{13}}
\end{align}
\end{document}
